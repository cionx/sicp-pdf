\subsection{Constructing Procedures Using \code{lambda}}
\label{Section 1.3.2}

In using \code{sum} as in \link{Section 1.3.1}, it seems terribly awkward to have to define trivial procedures such as \code{pi-term} and \code{pi-next} just so we can use them as arguments to our higher-order procedure.
Rather than define \code{pi-next} and \code{pi-term}, it would be more convenient to have a way to directly specify “the procedure that returns its input incremented by \( 4 \)” and “the procedure that returns the reciprocal of its input times its input plus \( 2 \).”
We can do this by introducing the special form \code{lambda}, which creates procedures.
Using \code{lambda} we can describe what we want as
\begin{scheme}
  (lambda (x) (+ x 4))
\end{scheme}
and
\begin{scheme}
  (lambda (x) (/ 1.0 (* x (+ x 2))))
\end{scheme}
Then our \code{pi-sum} procedure can be expressed without defining any auxiliary procedures as
\begin{scheme}
  (define (pi-sum a b)
    (sum (lambda (x) (/ 1.0 (* x (+ x 2))))
         a
         (lambda (x) (+ x 4))
         b))
\end{scheme}

Again using \code{lambda}, we can write the \code{integral} procedure without having to define the auxiliary procedure \code{add-dx}:
\begin{scheme}
  (define (integral f a b dx)
    (* (sum f
            (+ a (/ dx 2.0))
            (lambda (x) (+ x dx))
            b)
       dx))
\end{scheme}

In general, \code{lambda} is used to create procedures in the same way as \code{define}, except that no name is specified for the procedure:
\begin{scheme}
  (lambda (~\cproc{⟨formal-parameters⟩}~) ~\cvar{⟨body⟩}~)
\end{scheme}
The resulting procedure is just as much a procedure as one that is created using \code{define}.
The only difference is that it has not been associated with any name in the environment.
In fact,
\begin{scheme}
  (define (plus4 x) (+ x 4))
\end{scheme}
is equivalent to
\begin{scheme}
  (define plus4 (lambda (x) (+ x 4)))
\end{scheme}
We can read a \code{lambda} expression as follows:
\begin{scheme}
     (lambda                   (x)      (+     x     4))
~      |                         |        |     |     |~
~  the procedure  of an argument x  that  adds  x and 4~
\end{scheme}

Like any expression that has a procedure as its value, a \code{lambda} expression can be used as the operator in a combination such as
\begin{scheme}
  ((lambda (x y z) (+ x y (square z)))
   1 2 3)
  ~\outprint{12}~
\end{scheme}
or, more generally, in any context where we would normally use a procedure name.%
\footnote{
	It would be clearer and less intimidating to people learning Lisp if a name more obvious than \code{lambda}, such as \code{make-procedure}, were used.
	But the convention is firmly entrenched.
	The notation is adopted from the λ-calculus, a mathematical formalism introduced by the mathematical logician Alonzo \link{Church (1941)}.
	Church developed the λ-calculus to provide a rigorous foundation for studying the notions of function and function application.
	The λ-calculus has become a basic tool for mathematical investigations of the semantics of programming languages.
}



\subsubsection*{Using \code{let} to create local variables}

Another use of \code{lambda} is in creating local variables.
We often need local variables in our procedures other than those that have been bound as formal parameters.
For example, suppose we wish to compute the function
\[
  f(x,y) = x (1 + xy)^2 + y (1 - y) + (1 + xy) (1 - y) \,,
\]
which we could also express as
\begin{align*}
	a       &= 1 + x y  \,,         \\
	b       &= 1 - y    \,,         \\
	f(x, y) &= x a^2 + y b + a b \,.
\end{align*}
In writing a procedure to compute \( f \), we would like to include as local variables not only \( x \) and \( y \) but also the names of intermediate quantities like \( a \) and \( b \).
One way to accomplish this is to use an auxiliary procedure to bind the local variables:
\begin{scheme}
  (define (f x y)
    (define (f-helper a b)
      (+ (* x (square a))
         (* y b)
         (* a b)))
    (f-helper (+ 1 (* x y))
              (- 1 y)))
\end{scheme}

Of course, we could use a \code{lambda} expression to specify an anonymous procedure for binding our local variables.
The body of \code{f} then becomes a single call to that procedure:
\begin{scheme}
  (define (f x y)
    ((lambda (a b)
       (+ (* x (square a))
          (* y b)
          (* a b)))
     (+ 1 (* x y))
     (- 1 y)))
\end{scheme}
This construct is so useful that there is a special form called \code{let} to make its use more convenient.
Using \code{let}, the \code{f} procedure could be written as
\begin{scheme}
  (define (f x y)
    (let ((a (+ 1 (* x y)))
          (b (- 1 y)))
      (+ (* x (square a))
         (* y b)
         (* a b))))
\end{scheme}
The general form of a \code{let} expression is
\begin{scheme}
  (let ((~\cvar{⟨var\ind{1}⟩}~ ~\cvar{⟨exp\ind{1}⟩}~)
        (~\cvar{⟨var\ind{2}⟩}~ ~\cvar{⟨exp\ind{2}⟩}~)
        ~⋮~
        (~\cvar{⟨var\ind{n}⟩}~ ~\cvar{⟨exp\ind{n}⟩}~))
     ~\cvar{⟨body⟩}~)
\end{scheme}
which can be thought of as saying
\begin{scheme}
  ~let \var{⟨var\ind{1}⟩} have the value \var{⟨exp\ind{1}⟩} and~
      ~\var{⟨var\ind{2}⟩} have the value \var{⟨exp\ind{2}⟩} and~
      ~⋮~
      ~\var{⟨var\ind{n}⟩} have the value \var{⟨exp\ind{n}⟩}~
  ~in~  ~\var{⟨body⟩}~
\end{scheme}
The first part of the \code{let} expression is a list of name-expression pairs.
When the \code{let} is evaluated, each name is associated with the value of the corresponding expression.
The body of the \code{let} is evaluated with these names bound as local variables.
The way this happens is that the \code{let} expression is interpreted as an alternate syntax for
\begin{scheme}
  ((lambda (~\cvar{⟨var\ind{1}⟩}~ ~…~ ~\cvar{⟨var\ind{n}⟩}~)
      ~\cvar{⟨body⟩}~)
   ~\cvar{⟨exp\ind{1}⟩}~
   ~⋮~
   ~\cvar{⟨exp\ind{n}⟩}~)
\end{scheme}
No new mechanism is required in the interpreter in order to provide local variables.
A \code{let} expression is simply syntactic sugar for the underlying \code{lambda} application.

We can see from this equivalence that the scope of a variable specified by a \code{let} expression is the body of the \code{let}.
This implies that:

\begin{itemize}

	\item
		\code{let} allows one to bind variables as locally as possible to where they are to be used.
		For example, if the value of \code{x} is \( 5 \), the value of the expression
		\begin{scheme}
		  (+ (let ((x 3))
		       (+ x (* x 10)))
		     x)
		\end{scheme}
		is \( 38 \).
		Here, the \code{x} in the body of the \code{let} is \( 3 \), so the value of the \code{let} expression is \( 33 \).
		On the other hand, the \code{x} that is the second argument to the outermost \code{+} is still \( 5 \).

	\item
		The variables’ values are computed outside the \code{let}.
		This matters when the expressions that provide the values for the local variables depend upon variables having the same names as the local variables themselves.
		For example, if the value of \code{x} is \( 2 \), the expression
		\begin{scheme}
		  (let ((x 3)
		        (y (+ x 2)))
		    (* x y))
		\end{scheme}
		will have the value \( 12 \) because, inside the body of the \code{let}, \code{x} will be \( 3 \) and \code{y} will be \( 4 \) (which is the outer \code{x} plus \( 2 \)).

\end{itemize}

Sometimes we can use internal definitions to get the same effect as with \code{let}.
For example, we could have defined the procedure \code{f} above as
\begin{scheme}
  (define (f x y)
    (define a (+ 1 (* x y)))
    (define b (- 1 y))
    (+ (* x (square a))
       (* y b)
       (* a b)))
\end{scheme}
We prefer, however, to use \code{let} in situations like this and to use internal \code{define} only for internal procedures.%
\footnote{
	Understanding internal definitions well enough to be sure a program means what we intend it to mean requires a more elaborate model of the evaluation process than we have presented in this chapter.
	The subtleties do not arise with internal definitions of procedures, however.
	We will return to this issue in \link{Section 4.1.6}, after we learn more about evaluation.
}



\begin{exercise}
	\label{Exercise 1.34}
	Suppose we define the procedure
	\begin{scheme}
	  (define (f g)
	    (g 2))
	\end{scheme}
	Then we have
	\begin{scheme}
	  (f square)
	  ~\outprint{4}~

	  (f (lambda (z) (* z (+ z 1))))
	  ~\outprint{6}~
	\end{scheme}
	What happens if we (perversely) ask the interpreter to evaluate the combination \code{(f f)}?
	Explain.
\end{exercise}
