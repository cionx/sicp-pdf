\chapter*{References}
\addcontentsline{toc}{chapter}{References}
\label{References}

\fontspec[Numbers={OldStyle},StylisticSet=6]{Libertinus Serif}

\phantomsection \label{Abelson et al. 1992}
\textbf{Abelson}, Harold, Andrew Berlin, Jacob Katzenelson, William McAllister,
Guillermo Rozas, Gerald Jay Sussman, and Jack Wisdom. 1992.  The Supercomputer
Toolkit: A general framework for special-purpose computing.
\textit{International Journal of High-Speed Electronics} 3(3): 337--361.
\href{http://www.hpl.hp.com/techreports/94/HPL-94-30.html}{–›}

\phantomsection \label{Allen 1978}
\textbf{Allen}, John.  1978.  \textit{Anatomy of Lisp}. New York: McGraw-Hill.

\phantomsection \label{ANSI 1994}
\acronym{\textbf{ANSI}} \acronym{X}3.226-1994. \textit{American National Standard for Information
Sys\-tems---Programming Language---Common Lisp}.

\phantomsection \label{Appel 1987}
\textbf{Appel}, Andrew W.  1987.  Garbage collection can be faster than stack
allocation.  \textit{Information Processing Letters} 25(4): 275--279.
\href{https://www.cs.princeton.edu/~appel/papers/45.ps}{–›}

\phantomsection \label{Backus 1978}
\textbf{Backus}, John.  1978.  Can programming be liberated from the von Neumann style?
\textit{Communications of the \acronym{ACM}} 21(8): 613--641.
\href{http://worrydream.com/refs/Backus-CanProgrammingBeLiberated.pdf}{–›}

\phantomsection \label{Baker (1978)}
\textbf{Baker}, Henry G., Jr.  1978.  List processing in real time on a serial computer.
\textit{Communications of the \acronym{ACM}} 21(4): 280--293.
\href{http://dspace.mit.edu/handle/1721.1/41976}{–›}

\phantomsection \label{Batali et al. 1982}
\textbf{Batali}, John, Neil Mayle, Howard Shrobe, Gerald Jay Sussman, and Daniel Weise.
1982.  The Scheme-81 architecture---System and chip.  In \textit{Proceedings of
the \acronym{MIT} Conference on Advanced Research in \acronym{VLSI}}, edited by
Paul Penfield, Jr. Dedham, \acronym{MA}: Artech House.

\phantomsection \label{Borning (1977)}
\textbf{Borning}, Alan.  1977.  ThingLab---An object-oriented system for building
simulations using constraints. In \textit{Proceedings of the 5th International
Joint Conference on Artificial Intelligence}.
\href{http://ijcai.org/Past\%20Proceedings/IJCAI-77-VOL1/PDF/085.pdf}{–›}

\phantomsection \label{Borodin and Munro (1975)}
\textbf{Borodin}, Alan, and Ian Munro.  1975.  \textit{The Computational Complexity of
Algebraic and Numeric Problems}. New York: American Elsevier.

\phantomsection \label{Chaitin 1975}
\textbf{Chaitin}, Gregory J.  1975.  Randomness and mathematical proof.
\textit{Scientific American} 232(5): 47--52.
\href{https://www.cs.auckland.ac.nz/~chaitin/sciamer.html}{–›}

\phantomsection \label{Church (1941)}
\textbf{Church}, Alonzo.  1941.  \textit{The Calculi of Lambda-Conversion}.  Princeton,
N.J.: Princeton University Press.

\phantomsection \label{Clark (1978)}
\textbf{Clark}, Keith L.  1978.  Negation as failure.  In \textit{Logic and Data Bases}.
New York: Plenum Press, pp. 293--322.
\href{http://www.doc.ic.ac.uk/~klc/neg.html}{–›}

\phantomsection \label{Clinger (1982)}
\textbf{Clinger}, William.  1982.  Nondeterministic call by need is neither lazy nor by
name. In \textit{Proceedings of the \acronym{ACM} Symposium on Lisp and
Functional Programming}, pp. 226--234.

\phantomsection \label{Clinger and Rees 1991}
\textbf{Clinger}, William, and Jonathan Rees.  1991.  Macros that work.  In
\textit{Proceedings of the 1991 \acronym{ACM} Conference on Principles of
Programming Languages}, pp. 155--162.
\href{http://mumble.net/~jar/pubs/macros_that_work.ps}{–›}

\phantomsection \label{Colmerauer et al. 1973}
\textbf{Colmerauer} A., H. Kanoui, R. Pasero, and P. Roussel.  1973.  Un syst\`eme de
communication homme-machine en fran\c{c}ais.  Technical report, Groupe
Intelligence Artificielle, Universit\'e d'Aix Marseille, Luminy.
\href{http://alain.colmerauer.free.fr/alcol/ArchivesPublications/HommeMachineFr/HoMa.pdf}{–›}

\phantomsection \label{Cormen et al. 1990}
\textbf{Cormen}, Thomas, Charles Leiserson, and Ronald Rivest.  1990. \textit{Introduction
to Algorithms}. Cambridge, \acronym{MA}: \acronym{MIT} Press.

\phantomsection \label{Darlington et al. 1982}
\textbf{Darlington}, John, Peter Henderson, and David Turner.  1982.  \textit{Functional
Programming and Its Applications}. New York: Cambridge University Press.

\phantomsection \label{Dijkstra 1968a}
\textbf{Dijkstra}, Edsger W. 1968a.  The structure of the “\acronym{THE}”
multiprogramming system.  \textit{Communications of the \acronym{ACM}}
11(5): 341--346.
\href{http://www.cs.utexas.edu/users/EWD/ewd01xx/EWD196.PDF}{–›}

\phantomsection \label{Dijkstra 1968b}
\textbf{Dijkstra}, Edsger W. 1968b.  Cooperating sequential processes.  In
\textit{Programming Languages}, edited by F. Genuys. New York: Academic Press,
pp.  43--112.
\href{http://www.cs.utexas.edu/users/EWD/ewd01xx/EWD123.PDF}{–›}

\phantomsection \label{Dinesman 1968}
\textbf{Dinesman}, Howard P.  1968.  \textit{Superior Mathematical Puzzles}.  New York:
Simon and Schuster.

\phantomsection \label{deKleer et al. 1977}
\textbf{deKleer}, Johan, Jon Doyle, Guy Steele, and Gerald J. Sussman.  1977.
\acronym{AMORD}: Explicit control of reasoning.  In \textit{Proceedings of the
\acronym{ACM} Symposium on Artificial Intelligence and Programming Languages},
pp.  116--125.
\href{http://dspace.mit.edu/handle/1721.1/5750}{–›}

\phantomsection \label{Doyle (1979)}
\textbf{Doyle}, Jon. 1979. A truth maintenance system. \textit{Artificial Intelligence}
12: 231--272.
\href{http://dspace.mit.edu/handle/1721.1/5733}{–›}

\phantomsection \label{Feigenbaum and Shrobe 1993}
\textbf{Feigenbaum}, Edward, and Howard Shrobe. 1993. The Japanese National Fifth
Generation Project: Introduction, survey, and evaluation.  In \textit{Future
Generation Computer Systems}, vol. 9, pp. 105--117.
\href{https://saltworks.stanford.edu/assets/kv359wz9060.pdf}{–›}

\phantomsection \label{Feeley (1986)}
\textbf{Feeley}, Marc.  1986.  Deux approches \`a l'implantation du language
Scheme.  Masters thesis, Universit\'e de Montr\'eal.
\href{http://www.iro.umontreal.ca/~feeley/papers/FeeleyMSc.pdf}{–›}

\phantomsection \label{Feeley and Lapalme 1987}
\textbf{Feeley}, Marc and Guy Lapalme.  1987.  Using closures for code generation.
\textit{Journal of Computer Languages} 12(1): 47--66.
\href{http://citeseerx.ist.psu.edu/viewdoc/summary?doi=10.1.1.90.6978}{–›}

\textbf{Feller}, William.  1957.  \textit{An Introduction to Probability Theory and Its
Applications}, volume 1. New York: John Wiley \& Sons.

\phantomsection \label{Fenichel and Yochelson (1969)}
\textbf{Fenichel}, R., and J. Yochelson.  1969.  A Lisp garbage collector for virtual
memory computer systems.  \textit{Communications of the \acronym{ACM}}
12(11): 611--612.
\href{https://www.cs.purdue.edu/homes/hosking/690M/p611-fenichel.pdf}{–›}

\phantomsection \label{Floyd (1967)}
\textbf{Floyd}, Robert. 1967. Nondeterministic algorithms. \textit{\acronym{JACM}},
14(4): 636--644.
\href{http://citeseerx.ist.psu.edu/viewdoc/summary?doi=10.1.1.332.36}{–›}

\phantomsection \label{Forbus and deKleer 1993}
\textbf{Forbus}, Kenneth D., and Johan deKleer.  1993. \textit{Building Problem
Solvers}. Cambridge, \acronym{MA}: \acronym{MIT} Press.

\phantomsection \label{Friedman and Wise (1976)}
\textbf{Friedman}, Daniel P., and David S. Wise.  1976.  \acronym{CONS} should not
evaluate its arguments. In \textit{Automata, Languages, and Programming: Third
International Colloquium}, edited by S. Michaelson and R.  Milner, pp. 257--284.
\href{https://www.cs.indiana.edu/cgi-bin/techreports/TRNNN.cgi?trnum=TR44}{–›}

\phantomsection \label{Friedman et al. 1992}
\textbf{Friedman}, Daniel P., Mitchell Wand, and Christopher T. Haynes. 1992.
\textit{Essentials of Programming Languages}.  Cambridge, \acronym{MA}: \acronym{MIT}
Press/ McGraw-Hill.

\phantomsection \label{Gabriel 1988}
\textbf{Gabriel}, Richard P. 1988.  The Why of \emph{Y}.  \textit{Lisp Pointers}
2(2): 15--25.
\href{http://www.dreamsongs.com/Files/WhyOfY.pdf}{–›}

\textbf{Goldberg}, Adele, and David Robson.  1983.  \textit{Smalltalk-80: The Language and
Its Implementation}. Reading, \acronym{MA}: Addison-Wesley.
\href{http://stephane.ducasse.free.fr/FreeBooks/BlueBook/Bluebook.pdf}{–›}

\phantomsection \label{Gordon et al. 1979}
\textbf{Gordon}, Michael, Robin Milner, and Christopher Wadsworth.  1979.
\textit{Edinburgh \acronym{LCF}}. Lecture Notes in Computer Science, volume 78. New York:
Springer-Verlag.

\phantomsection \label{Gray and Reuter 1993}
\textbf{Gray}, Jim, and Andreas Reuter. 1993. \textit{Transaction Processing: Concepts and
Models}. San Mateo, \acronym{CA}: Morgan-Kaufman.

\phantomsection \label{Green 1969}
\textbf{Green}, Cordell.  1969.  Application of theorem proving to problem solving.  In
\textit{Proceedings of the International Joint Conference on Artificial
Intelligence}, pp. 219--240.
\href{http://citeseer.ist.psu.edu/viewdoc/summary?doi=10.1.1.81.9820}{–›}

\phantomsection \label{Green and Raphael (1968)}
\textbf{Green}, Cordell, and Bertram Raphael.  1968.  The use of theorem-proving
techniques in question-answering systems.  In \textit{Proceedings of the
\acronym{ACM} National Conference}, pp. 169--181.
\href{http://www.kestrel.edu/home/people/green/publications/green-raphael.pdf}{–›}

\phantomsection \label{Griss 1981}
\textbf{Griss}, Martin L.  1981.  Portable Standard Lisp, a brief overview.  Utah
Symbolic Computation Group Operating Note 58, University of Utah.

\phantomsection \label{Guttag 1977}
\textbf{Guttag}, John V.  1977.  Abstract data types and the development of data
structures.  \textit{Communications of the \acronym{ACM}} 20(6): 396--404.
\href{http://www.unc.edu/~stotts/comp723/guttagADT77.pdf}{–›}

\phantomsection \label{Hamming 1980}
\textbf{Hamming}, Richard W.  1980.  \textit{Coding and Information Theory}.  Englewood
Cliffs, N.J.: Prentice-Hall.

\phantomsection \label{Hanson 1990}
\textbf{Hanson}, Christopher P.  1990.  Efficient stack allocation for tail-recur\-sive
languages.  In \textit{Proceedings of \acronym{ACM} Conference on Lisp and
Functional Programming}, pp. 106--118.
\href{https://groups.csail.mit.edu/mac/ftpdir/users/cph/links.ps.gz}{–›}

\phantomsection \label{Hanson 1991}
\textbf{Hanson}, Christopher P.  1991.  A syntactic closures macro facility.  \textit{Lisp
Pointers}, 4(3).
\href{http://groups.csail.mit.edu/mac/ftpdir/scheme-reports/synclo.ps}{–›}

\phantomsection \label{Hardy 1921}
\textbf{Hardy}, Godfrey H.  1921.  Srinivasa Ramanujan.  \textit{Proceedings of the London
Mathematical Society} \acronym{XIX}(2).

\phantomsection \label{Hardy and Wright 1960}
\textbf{Hardy}, Godfrey H., and E. M. Wright.  1960.  \textit{An Introduction to the
Theory of Numbers}.  4th edition.  New York: Oxford University Press.
\href{https://archive.org/details/AnIntroductionToTheTheoryOfNumbers-4thEd-G.h.HardyE.m.Wright}{–›}

\phantomsection \label{Havender (1968)}
\textbf{Havender}, J. 1968. Avoiding deadlocks in multi-tasking systems. \textit{\acronym{IBM}
Systems Journal} 7(2): 74--84.

\phantomsection \label{Hearn 1969}
\textbf{Hearn}, Anthony C.  1969.  Standard Lisp.  Technical report \acronym{AIM}-90,
Artificial Intelligence Project, Stanford University.
\href{http://www.softwarepreservation.org/projects/LISP/stanford/Hearn-StandardLisp-AIM-90.pdf}{–›}

\phantomsection \label{Henderson 1980}
\textbf{Henderson}, Peter. 1980.  \textit{Functional Programming: Application and
Implementation}. Englewood Cliffs, N.J.: Prentice-Hall.

\phantomsection \label{Henderson 1982}
\textbf{Henderson}, Peter. 1982. Functional Geometry. In \textit{Conference Record of the
1982 \acronym{ACM} Symposium on Lisp and Functional Programming}, pp. 179--187.
\href{http://pmh-systems.co.uk/phAcademic/papers/funcgeo.pdf}{–›},
\href{http://eprints.soton.ac.uk/257577/1/funcgeo2.pdf}{2002 version –›}

\phantomsection \label{Hewitt (1969)}
\textbf{Hewitt}, Carl E.  1969.  \acronym{PLANNER}: A language for proving
theorems in robots.  In \textit{Proceedings of the International Joint
Conference on Artificial Intelligence}, pp. 295--301.
\href{http://dspace.mit.edu/handle/1721.1/6171}{–›}

\phantomsection \label{Hewitt (1977)}
\textbf{Hewitt}, Carl E.  1977.  Viewing control structures as patterns of passing
messages.  \textit{Journal of Artificial Intelligence} 8(3): 323--364.
\href{http://dspace.mit.edu/handle/1721.1/6272}{–›}

\phantomsection \label{Hoare (1972)}
\textbf{Hoare}, C. A. R. 1972.  Proof of correctness of data representations.
\textit{Acta Informatica} 1(1).

\phantomsection \label{Hodges 1983}
\textbf{Hodges}, Andrew. 1983.  \textit{Alan Turing: The Enigma}. New York: Simon and
Schuster.

\phantomsection \label{Hofstadter 1979}
\textbf{Hofstadter}, Douglas R.  1979.  \textit{G\"odel, Escher, Bach: An Eternal Golden
Braid}. New York: Basic Books.

\phantomsection \label{Hughes 1990}
\textbf{Hughes}, R. J. M.  1990.  Why functional programming matters.  In \textit{Research
Topics in Functional Programming}, edited by David Turner.  Reading, \acronym{MA}:
Addison-Wesley, pp. 17--42.
\href{http://www.cs.kent.ac.uk/people/staff/dat/miranda/whyfp90.pdf}{–›}

\phantomsection \label{IEEE 1990}
\acronym{\textbf{IEEE}} Std 1178-1990.  1990.  \textit{\acronym{IEEE} Standard for the
Scheme Programming Language}.

\enlargethispage{\baselineskip}

\phantomsection \label{Ingerman et al. 1960}
\textbf{Ingerman}, Peter, Edgar Irons, Kirk Sattley, and Wallace Feurzeig; assisted by
M. Lind, Herbert Kanner, and Robert Floyd.  1960.  \acronym{THUNKS}: A way of
compiling procedure statements, with some comments on procedure declarations.
Unpublished manuscript.  (Also, private communication from Wallace Feurzeig.)

\phantomsection \label{Kaldewaij 1990}
\textbf{Kaldewaij}, Anne. 1990.  \textit{Programming: The Derivation of Algorithms}. New
York: Prentice-Hall.

\phantomsection \label{Knuth (1973)}
\textbf{Knuth}, Donald E.  1973.  \textit{Fundamental Algorithms}. Volume 1 of \textit{The
Art of Computer Programming}.  2nd edition. Reading, \acronym{MA}: Addison-Wesley.

\phantomsection \label{Knuth 1981}
\textbf{Knuth}, Donald E.  1981.  \textit{Seminumerical Algorithms}. Volume 2 of \textit{The
Art of Computer Programming}.  2nd edition. Reading, \acronym{MA}: Addison-Wesley.

\phantomsection \label{Kohlbecker 1986}
\textbf{Kohlbecker}, Eugene Edmund, Jr. 1986.  Syntactic extensions in the programming
language Lisp.  Ph.D. thesis, Indiana University.
\href{http://www.ccs.neu.edu/scheme/pubs/dissertation-kohlbecker.pdf}{–›}

\phantomsection \label{Konopasek and Jayaraman 1984}
\textbf{Konopasek}, Milos, and Sundaresan Jayaraman.  1984.  \textit{The TK!Solver Book: A
Guide to Problem-Solving in Science, Engineering, Business, and
Education}. Berkeley, \acronym{CA}: Osborne/McGraw-Hill.

\phantomsection \label{Kowalski (1973; 1979)}
\textbf{Kowalski}, Robert.  1973.  Predicate logic as a programming language.  Technical
report 70, Department of Computational Logic, School of Artificial
Intelligence, University of Edinburgh.
\href{http://www.doc.ic.ac.uk/~rak/papers/IFIP\%2074.pdf}{–›}

\textbf{Kowalski}, Robert.  1979.  \textit{Logic for Problem Solving}. New York:
North-Holland.
\href{http://www.doc.ic.ac.uk/\%7Erak/papers/LogicForProblemSolving.pdf}{–›}

\phantomsection \label{Lamport (1978)}
\textbf{Lamport}, Leslie. 1978.  Time, clocks, and the ordering of events in a
distributed system.  \textit{Communications of the \acronym{ACM}} 21(7): 558--565.
\href{http://research.microsoft.com/en-us/um/people/lamport/pubs/time-clocks.pdf}{–›}

\phantomsection \label{Lampson et al. 1981}
\textbf{Lampson}, Butler, J. J. Horning, R.  London, J. G. Mitchell, and G. K.  Popek.
1981.  Report on the programming language Euclid.  Technical report, Computer
Systems Research Group, University of Toronto.
\href{http://www.bitsavers.org/pdf/xerox/parc/techReports/CSL-81-12_Report_On_The_Programming_Language_Euclid.pdf}{–›}

\phantomsection \label{Landin (1965)}
\textbf{Landin}, Peter.  1965.  A correspondence between Algol~60 and Church’s lambda
notation: Part I.  \textit{Communications of the \acronym{ACM}} 8(2): 89--101.

\phantomsection \label{Lieberman and Hewitt 1983}
\textbf{Lieberman}, Henry, and Carl E. Hewitt. 1983. A real-time garbage collector based
on the lifetimes of objects. \textit{Communications of the \acronym{ACM}}
26(6): 419--429.
\href{http://dspace.mit.edu/handle/1721.1/6335}{–›}

\phantomsection \label{Liskov and Zilles (1975)}
\textbf{Liskov}, Barbara H., and Stephen N. Zilles.  1975.  Specification techniques for
data abstractions.  \textit{\acronym{IEEE} Transactions on Software Engineering}
1(1): 7--19.
\href{http://csg.csail.mit.edu/CSGArchives/memos/Memo-117.pdf}{–›}

\phantomsection \label{McAllester (1978; 1980)}
\textbf{McAllester}, David Allen.  1978.  A three-valued truth-maintenance system.  Memo
473, \acronym{MIT} Artificial Intelligence Laboratory.
\href{http://dspace.mit.edu/handle/1721.1/6296}{–›}

\textbf{McAllester}, David Allen.  1980.  An outlook on truth maintenance.  Memo 551,
\acronym{MIT} Artificial Intelligence Laboratory.
\href{http://dspace.mit.edu/handle/1721.1/6327}{–›}

\phantomsection \label{McCarthy 1960}
\textbf{McCarthy}, John.  1960.  Recursive functions of symbolic expressions and their
computation by machine.  \textit{Communications of the \acronym{ACM}}
3(4): 184--195.
\href{http://www-formal.stanford.edu/jmc/recursive.pdf}{–›}

\phantomsection \label{McCarthy 1963}
\textbf{McCarthy}, John.  1963.  A basis for a mathematical theory of computation.  In
\textit{Computer Programming and Formal Systems}, edited by P. Braffort and
D. Hirschberg.  North-Holland.
\href{http://www-formal.stanford.edu/jmc/basis.html}{–›}

\phantomsection \label{McCarthy 1978}
\textbf{McCarthy}, John.  1978.  The history of Lisp.  In \textit{Proceedings of the
\acronym{ACM} \acronym{SIGPLAN} Conference on the History of Programming
Languages}.
\href{http://www-formal.stanford.edu/jmc/history/lisp/lisp.html}{–›}

\phantomsection \label{McCarthy et al. 1965}
\textbf{McCarthy}, John, P. W. Abrahams, D. J. Edwards, T. P. Hart, and M. I.  Levin.
1965.  \textit{Lisp 1.5 Programmer’s Manual}.  2nd edition.  Cambridge, \acronym{MA}:
\acronym{MIT} Press.
\href{http://www.softwarepreservation.org/projects/LISP/book/LISP\%201.5\%20Programmers\%20Manual.pdf/view}{–›}

\phantomsection \label{McDermott and Sussman (1972)}
\textbf{McDermott}, Drew, and Gerald Jay Sussman.  1972. Conniver reference manual.
Memo 259, \acronym{MIT} Artificial Intelligence Laboratory.
\href{http://dspace.mit.edu/handle/1721.1/6203}{–›}

\phantomsection \label{Miller 1976}
\textbf{Miller}, Gary L.  1976.  Riemann’s Hypothesis and tests for primality.
\textit{Journal of Computer and System Sciences} 13(3): 300--317.
\href{http://www.cs.cmu.edu/~glmiller/Publications/b2hd-Mi76.html}{–›}

\phantomsection \label{Miller and Rozas 1994}
\textbf{Miller}, James S., and Guillermo J. Rozas. 1994.  Garbage collection is fast,
but a stack is faster.  Memo 1462, \acronym{MIT} Artificial Intelligence
Laboratory.
\href{http://dspace.mit.edu/handle/1721.1/6622}{–›}

\phantomsection \label{Moon 1978}
\textbf{Moon}, David.  1978.  MacLisp reference manual, Version 0.  Technical report,
\acronym{MIT} Laboratory for Computer Science.
\href{http://www.softwarepreservation.org/projects/LISP/MIT/Moon-MACLISP_Reference_Manual-Apr_08_1974.pdf/view}{–›}

\phantomsection \label{Moon and Weinreb 1981}
\textbf{Moon}, David, and Daniel Weinreb.  1981.  Lisp machine manual.  Technical
report, \acronym{MIT} Artificial Intelligence Laboratory.
\href{http://www.unlambda.com/lmman/index.html}{–›}

\enlargethispage{\baselineskip}

\phantomsection \label{Morris et al. 1980}
\textbf{Morris}, J. H., Eric Schmidt, and Philip Wadler.  1980.  Experience with an
applicative string processing language.  In \textit{Proceedings of the 7th Annual
\acronym{ACM} \acronym{SIGACT}/\acronym{SIGPLAN} Symposium on the Principles of
Programming Languages}.

\phantomsection \label{Phillips 1934}
\textbf{Phillips}, Hubert.  1934. \textit{The Sphinx Problem Book}.  London: Faber and
Faber.

\phantomsection \label{Pitman 1983}
\textbf{Pitman}, Kent. 1983. The revised MacLisp Manual (Saturday evening edition).
Technical report 295, \acronym{MIT} Laboratory for Computer Science.
\href{http://maclisp.info/pitmanual}{–›}

\phantomsection \label{Rabin 1980}
\textbf{Rabin}, Michael O. 1980. Probabilistic algorithm for testing primality.
\textit{Journal of Number Theory} 12: 128--138.

\phantomsection \label{Raymond 1993}
\textbf{Raymond}, Eric.  1993. \textit{The New Hacker’s Dictionary}. 2nd edition.
Cambridge, \acronym{MA}: \acronym{MIT} Press.
\href{http://www.catb.org/jargon/}{–›}

\textbf{Raynal}, Michel. 1986. \textit{Algorithms for Mutual Exclusion}.  Cambridge, \acronym{MA}:
\acronym{MIT} Press.

\phantomsection \label{Rees and Adams 1982}
\textbf{Rees}, Jonathan A., and Norman I. Adams \acronym{IV}. 1982.  T: A dialect of Lisp or,
lambda: The ultimate software tool.  In \textit{Conference Record of the 1982
\acronym{ACM} Symposium on Lisp and Functional Programming}, pp.  114--122.
\href{http://people.csail.mit.edu/riastradh/t/adams82t.pdf}{–›}

\textbf{Rees}, Jonathan, and William Clinger (eds). 1991.  The revised\textsuperscript{4} report on the algorithmic language Scheme.  \textit{Lisp Pointers}, 4(3).
\href{http://people.csail.mit.edu/jaffer/r4rs.pdf}{–›}

\phantomsection \label{Rivest et al. (1977)}
\textbf{Rivest}, Ronald, Adi Shamir, and Leonard Adleman.  1977.  A method for obtaining
digital signatures and public-key cryptosystems. Technical memo \acronym{LCS}/\acronym{TM82},
\acronym{MIT} Laboratory for Computer Science.
\href{http://people.csail.mit.edu/rivest/Rsapaper.pdf}{–›}

\phantomsection \label{Robinson 1965}
\textbf{Robinson}, J. A. 1965.  A machine-oriented logic based on the resolution
principle.  \textit{Journal of the \acronym{ACM}} 12(1): 23.

\phantomsection \label{Robinson 1983}
\textbf{Robinson}, J. A. 1983.  Logic programming---Past, present, and future.
\textit{New Generation Computing} 1: 107--124.

\phantomsection \label{Spafford 1989}
\textbf{Spafford}, Eugene H.  1989.  The Internet Worm: Crisis and aftermath.
\textit{Communications of the \acronym{ACM}} 32(6): 678--688.
\href{http://citeseerx.ist.psu.edu/viewdoc/download?doi=10.1.1.123.8503&rep=rep1&type=pdf}{–›}

\phantomsection \label{Steele 1977}
\textbf{Steele}, Guy Lewis, Jr.  1977.  Debunking the “expensive procedure call” myth.
In \textit{Proceedings of the National Conference of the \acronym{ACM}},
pp. 153--62.
\href{http://dspace.mit.edu/handle/1721.1/5753}{–›}

\pagebreak

\phantomsection \label{Steele 1982}
\textbf{Steele}, Guy Lewis, Jr.  1982.  An overview of Common Lisp.  In
\textit{Proceedings of the \acronym{ACM} Symposium on Lisp and Functional
Programming}, pp. 98--107.

\phantomsection \label{Steele 1990}
\textbf{Steele}, Guy Lewis, Jr.  1990.  \textit{Common Lisp: The Language}. 2nd edition.
Digital Press.
\href{http://www.cs.cmu.edu/Groups/AI/html/cltl/cltl2.html}{–›}

\phantomsection \label{Steele and Sussman 1975}
\textbf{Steele}, Guy Lewis, Jr., and Gerald Jay Sussman.  1975.  Scheme: An interpreter
for the extended lambda calculus.  Memo 349, \acronym{MIT} Artificial
Intelligence Laboratory.
\href{http://dspace.mit.edu/handle/1721.1/5794}{–›}

\phantomsection \label{Steele et al. 1983}
\textbf{Steele}, Guy Lewis, Jr., Donald R. Woods, Raphael A. Finkel, Mark R.  Crispin,
Richard M. Stallman, and Geoffrey S. Goodfellow.  1983.  \textit{The Hacker’s
Dictionary}. New York: Harper \& Row.
\href{http://www.dourish.com/goodies/jargon.html}{–›}

\phantomsection \label{Stoy 1977}
\textbf{Stoy}, Joseph E.  1977.  \textit{Denotational Semantics}. Cambridge, \acronym{MA}:
\acronym{MIT} Press.

\phantomsection \label{Sussman and Stallman 1975}
\textbf{Sussman}, Gerald Jay, and Richard M. Stallman.  1975.  Heuristic techniques in
computer-aided circuit analysis.  \textit{\acronym{IEEE} Transactions on Circuits
and Systems} \acronym{CAS}-22(11): 857--865.
\href{http://dspace.mit.edu/handle/1721.1/5803}{–›}

\phantomsection \label{Sussman and Steele 1980}
\textbf{Sussman}, Gerald Jay, and Guy Lewis Steele Jr.  1980.  Constraints---A language
for expressing almost-hierachical descriptions.  \textit{AI Journal} 14: 1--39.
\href{http://dspace.mit.edu/handle/1721.1/6312}{–›}

\phantomsection \label{Sussman and Wisdom 1992}
\textbf{Sussman}, Gerald Jay, and Jack Wisdom.  1992. Chaotic evolution of the solar
system.  \textit{Science} 257: 256--262.
\href{http://groups.csail.mit.edu/mac/users/wisdom/ss-chaos.pdf}{–›}

\phantomsection \label{Sussman et al. (1971)}
\textbf{Sussman}, Gerald Jay, Terry Winograd, and Eugene Charniak.  1971.  Microplanner
reference manual.  Memo 203\acronym{A}, \acronym{MIT} Artificial Intelligence Laboratory.
\href{http://dspace.mit.edu/handle/1721.1/6184}{–›}

\phantomsection \label{Sutherland (1963)}
\textbf{Sutherland}, Ivan E.  1963.  \acronym{SKETCHPAD}: A man-machine graphical
communication system.  Technical report 296, \acronym{MIT} Lincoln Laboratory.
\href{https://www.cl.cam.ac.uk/techreports/UCAM-CL-TR-574.pdf}{–›}

\phantomsection \label{Teitelman 1974}
\textbf{Teitelman}, Warren.  1974.  Interlisp reference manual.  Technical report, Xerox
Palo Alto Research Center.
\href{http://www.softwarepreservation.org/projects/LISP/interlisp/Interlisp-Oct_1974.pdf/view}{–›}

\phantomsection \label{Thatcher et al. 1978}
\textbf{Thatcher}, James W., Eric G. Wagner, and Jesse B. Wright. 1978.  Data type
specification: Parameterization and the power of specification techniques. In
\textit{Conference Record of the Tenth Annual \acronym{ACM} Symposium on Theory
of Computing}, pp. 119--132.

\phantomsection \label{Turner 1981}
\textbf{Turner}, David.  1981.  The future of applicative languages.  In
\textit{Proceedings of the 3rd European Conference on Informatics}, Lecture Notes
in Computer Science, volume 123. New York: Springer-Verlag, pp.  334--348.

\phantomsection \label{Wand 1980}
\textbf{Wand}, Mitchell.  1980.  Continuation-based program transformation strategies.
\textit{Journal of the \acronym{ACM}} 27(1): 164--180.
\href{http://www.diku.dk/OLD/undervisning/2005e/224/papers/Wand80.pdf}{–›}

\phantomsection \label{Waters (1979)}
\textbf{Waters}, Richard C.  1979.  A method for analyzing loop programs.
\textit{\acronym{IEEE} Transactions on Software Engineering} 5(3): 237--247.

\textbf{Winograd}, Terry.  1971.  Procedures as a representation for data in a computer
program for understanding natural language.  Technical report \acronym{AI TR}-17,
\acronym{MIT} Artificial Intelligence Laboratory.
\href{http://dspace.mit.edu/handle/1721.1/7095}{–›}

\phantomsection \label{Winston 1992}
\textbf{Winston}, Patrick. 1992. \textit{Artificial Intelligence}.  3rd edition.  Reading,
\acronym{MA}: Addison-Wesley.

\phantomsection \label{Zabih et al. 1987}
\textbf{Zabih}, Ramin, David McAllester, and David Chapman.  1987.  Non-deterministic
Lisp with dependency-directed backtracking.  \textit{\acronym{AAAI}-87},
pp. 59--64.
\href{http://www.aaai.org/Papers/AAAI/1987/AAAI87-011.pdf}{–›}

\phantomsection \label{Zippel (1979)}
\textbf{Zippel}, Richard.  1979.  Probabilistic algorithms for sparse polynomials.
Ph.D. dissertation, Department of Electrical Engineering and Computer Science,
\acronym{MIT}.

\phantomsection \label{Zippel 1993}
\textbf{Zippel}, Richard.  1993.  \textit{Effective Polynomial Computation}.  Boston, \acronym{MA}:
Kluwer Academic Publishers.

\fontspec[Numbers={Lining}]{Libertinus Serif}
