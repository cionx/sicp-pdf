\subsection{Representing Expressions}
\label{Section 4.1.2}

The evaluator is reminiscent of the symbolic differentiation program discussed in \link{Section 2.3.2}.
Both programs operate on symbolic expressions.
In both programs, the result of operating on a compound expression is determined by operating recursively on the pieces of the expression and combining the results in a way that depends on the type of the expression.
In both programs we used data abstraction to decouple the general rules of operation from the details of how expressions are represented.
In the differentiation program this meant that the same differentiation procedure could deal with algebraic expressions in prefix form, in infix form, or in some other form.
For the evaluator, this means that the syntax of the language being evaluated is determined solely by the procedures that classify and extract pieces of expressions.

Here is the specification of the syntax of our language:
\begin{itemize}[leftmargin=*]

	\item
		The only self-evaluating items are numbers and strings:

		\begin{scheme}
		  (define (self-evaluating? exp)
		    (cond ((number? exp) true)
		          ((string? exp) true)
		          (else false)))
		\end{scheme}

	\item
		Variables are represented by symbols:

		\begin{scheme}
		  (define (variable? exp) (symbol? exp))
		\end{scheme}

	\item
		Quotations have the form \code{(quote ⟨\var{text-of-quotation}⟩)}:%
		\footnote{
			As mentioned in \link{Section 2.3.1}, the evaluator sees a quoted expression as a list beginning with \code{quote}, even if the expression is typed with the quotation mark.
			For example, the expression \code{'a} would be seen by the evaluator as \code{(quote a)}.
		See \link{Exercise 2.55}.
		}
		\begin{scheme}
		  (define (quoted? exp) (tagged-list? exp 'quote))
		  (define (text-of-quotation exp) (cadr exp))
		\end{scheme}
		\code{quoted?} is defined in terms of the procedure \code{tagged-list?}, which identifies lists beginning with a designated symbol:
		\begin{scheme}
		  (define (tagged-list? exp tag)
		    (if (pair? exp)
		        (eq? (car exp) tag)
		        false))
		\end{scheme}

	\item
		Assignments have the form \code{(set! ⟨\var{var}⟩ ⟨\var{value}⟩)}:

		\begin{scheme}
		  (define (assignment? exp) (tagged-list? exp 'set!))

		  (define (assignment-variable exp) (cadr exp))

		  (define (assignment-value exp) (caddr exp))
		\end{scheme}

	\item
		Definitions have the form
		\begin{scheme}
		  (define ⟨~\var{var}~⟩ ⟨~\var{value}~⟩)
		\end{scheme}
		or the form
		\begin{scheme}
		  (define (⟨~\var{var}~⟩ ⟨~\var{parameter}\ind{1}~⟩ … ⟨~\var{parameter}\ind{n}~⟩
		    ⟨~\var{body}~⟩)
		\end{scheme}
		The latter form (standard procedure definition) is syntactic sugar for
		\begin{scheme}
		  (define ⟨~\var{var}~⟩
		    (lambda (⟨~\var{parameter}\ind{1}~⟩ … ⟨~\var{parameter}\ind{n}~⟩)
		      ⟨~\var{body}~⟩))
		\end{scheme}
		The corresponding syntax procedures are the following:
		\begin{scheme}
		  (define (definition? exp) (tagged-list? exp 'define))

		  (define (definition-variable exp)
		    (if (symbol? (cadr exp))
		        (cadr exp)
		        (caadr exp)))

		  (define (definition-value exp)
		    (if (symbol? (cadr exp))
		        (caddr exp)
		        (make-lambda (cdadr exp)     ~\textrm{; formal parameters}~
		                     (cddr exp))))   ~\textrm{; body}~
		\end{scheme}

	\item
		\code{lambda} expressions are lists that begin with the symbol \code{lambda}:

	\begin{scheme}
	  (define (lambda? exp) (tagged-list? exp 'lambda))

	  (define (lambda-parameters exp) (cadr exp))

	  (define (lambda-body exp) (cddr exp))
	\end{scheme}
	We also provide a constructor for \code{lambda} expressions, which is used by
	\code{definition-value}, above:
	\begin{scheme}
	  (define (make-lambda parameters body)
	    (cons 'lambda (cons parameters body)))
	\end{scheme}

	\item
		Conditionals begin with \code{if} and have a predicate, a consequent, and an (optional) alternative.
		If the expression has no alternative part, we provide \code{false} as the alternative.%
		\footnote{
			The value of an \code{if} expression when the predicate is false and there is no alternative is unspecified in Scheme;
			we have chosen here to make it false.
			We will support the use of the variables \code{true} and \code{false} in expressions to be evaluated by binding them in the global environment.
			See \link{Section 4.1.4}.
		}

		\begin{scheme}
		  (define (if? exp) (tagged-list? exp 'if))

		  (define (if-predicate exp) (cadr exp))

		  (define (if-consequent exp) (caddr exp))

		  (define (if-alternative exp)
		    (if (not (null? (cdddr exp)))
		        (cadddr exp)
		        'false))
		\end{scheme}

		We also provide a constructor for \code{if} expressions, to be used by \code{cond->if} to transform \code{cond} expressions into \code{if} expressions:

		\begin{scheme}
		  (define (make-if predicate consequent alternative)
		    (list 'if predicate consequent alternative))
		\end{scheme}

	\item
		\code{begin} packages a sequence of expressions into a single expression.
		We include syntax operations on \code{begin} expressions to extract the actual sequence from the \code{begin} expression, as well as selectors that return the first expression and the rest of the expressions in the sequence.%
		\footnote{
			These selectors for a list of expressions---and the corresponding ones for a list of operands---are not intended as a data abstraction.
			They are introduced as mnemonic names for the basic list operations in order to make it easier to understand the explicit-control evaluator in \link{Section 5.4}.
		}

		\begin{scheme}
		  (define (begin? exp) (tagged-list? exp 'begin))

		  (define (begin-actions exp) (cdr exp))

		  (define (last-exp? seq) (null? (cdr seq)))

		  (define (first-exp seq) (car seq))

		  (define (rest-exps seq) (cdr seq))
		\end{scheme}

		We also include a constructor \code{sequence->exp} (for use by \code{cond->if}) that transforms a sequence into a single expression, using \code{begin} if necessary:

		\begin{scheme}
		  (define (sequence->exp seq)
		    (cond ((null? seq) seq)
		          ((last-exp? seq) (first-exp seq))
		          (else (make-begin seq))))

		  (define (make-begin seq) (cons 'begin seq))
		\end{scheme}

	\item
		A procedure application is any compound expression that is not one of the above expression types.
		The \code{car} of the expression is the operator, and the \code{cdr} is the list of operands:

		\begin{scheme}
		  (define (application? exp) (pair? exp))

		  (define (operator exp) (car exp))

		  (define (operands exp) (cdr exp))

		  (define (no-operands? ops) (null? ops))

		  (define (first-operand ops) (car ops))

		  (define (rest-operands ops) (cdr ops))
		\end{scheme}

\end{itemize}



\subsubsection*{Derived expressions}

Some special forms in our language can be defined in terms of expressions involving other special forms, rather than being implemented directly.
One example is \code{cond}, which can be implemented as a nest of \code{if} expressions.
For example, we can reduce the problem of evaluating the expression
\begin{scheme}
  (cond ((> x 0) x)
        ((= x 0) (display 'zero) 0)
        (else (- x)))
\end{scheme}
to the problem of evaluating the following expression involving \code{if} and \code{begin} expressions:
\begin{scheme}
  (if (> x 0)
      x
      (if (= x 0)
          (begin (display 'zero) 0)
          (- x)))
\end{scheme}
Implementing the evaluation of \code{cond} in this way simplifies the evaluator because it reduces the number of special forms for which the evaluation process must be explicitly specified.

We include syntax procedures that extract the parts of a \code{cond} expression, and a procedure \code{cond->if} that transforms \code{cond} expressions into \code{if} expressions.
A case analysis begins with \code{cond} and has a list of predicate-action clauses.
A clause is an \code{else} clause if its predicate is the symbol \code{else}.%
\footnote{
	The value of a \code{cond} expression when all the predicates are false and there is no \code{else} clause is unspecified in Scheme;
	we have chosen here to make it false.
}
\begin{scheme}
  (define (cond? exp) (tagged-list? exp 'cond))

  (define (cond-clauses exp) (cdr exp))

  (define (cond-else-clause? clause)
    (eq? (cond-predicate clause) 'else))

  (define (cond-predicate clause) (car clause))

  (define (cond-actions clause) (cdr clause))

  (define (cond->if exp) (expand-clauses (cond-clauses exp)))

  (define (expand-clauses clauses)
    (if (null? clauses)
        'false                        ~\textrm{; no \code{else} clause}~
        (let ((first (car clauses))
              (rest (cdr clauses)))
          (if (cond-else-clause? first)
              (if (null? rest)
                  (sequence->exp (cond-actions first))
                  (error "ELSE clause isn’t last: COND->IF"
                         clauses))
              (make-if (cond-predicate first)
                       (sequence->exp (cond-actions first))
                       (expand-clauses rest))))))
\end{scheme}

Expressions (such as \code{cond}) that we choose to implement as syntactic transformations are called \newterm{derived expressions}.
\code{let} expressions are also derived expressions (see
\link{Exercise 4.6}).%
\footnote{
	Practical Lisp systems provide a mechanism that allows a user to add new derived expressions and specify their implementation as syntactic transformations without modifying the evaluator.
	Such a user-defined transformation is called a \newterm{macro}.
	Although it is easy to add an elementary mechanism for defining macros, the resulting language has subtle name-conflict problems.
	There has been much research on mechanisms for macro definition that do not cause these difficulties.
	See, for example, \link{Kohlbecker 1986}, \link{Clinger and Rees 1991}, and \link{Hanson 1991}.
}



\begin{exercise}
	\label{Exercise 4.2}
	Louis Reasoner plans to reorder the \code{cond} clauses in \code{eval} so that the clause for procedure applications appears before the clause for assignments.
	He argues that this will make the interpreter more efficient:
	Since programs usually contain more applications than assignments, definitions, and so on, his modified \code{eval} will usually check fewer clauses than the original \code{eval} before identifying the type of an expression.
	\begin{enumerate}[label=\alph*., leftmargin = *]

		\item
			What is wrong with Louis’s plan?
			(Hint: What will Louis’s evaluator do with the expression \code{(define x 3)}?)

		\item
			Louis is upset that his plan didn’t work.
			He is willing to go to any lengths to make his evaluator recognize procedure applications before it checks for most other kinds of expressions.
			Help him by changing the syntax of the evaluated language so that procedure applications start with \code{call}.
			For example, instead of \code{(factorial 3)} we will now have to write \code{(call factorial 3)} and instead of \code{(+ 1 2)} we will have to write \code{(call + 1 2)}.

	\end{enumerate}
\end{exercise}



\begin{exercise}
	\label{Exercise 4.3}
	Rewrite \code{eval} so that the dispatch is done in data-directed style.
	Compare this with the data-directed differentiation procedure of \link{Exercise 2.73}.
	(You may use the \code{car} of a compound expression as the type of the expression, as is appropriate for the syntax implemented in this section.)
\end{exercise}



\begin{exercise}
	\label{Exercise 4.4}
	Recall the definitions of the
	special forms \code{and} and \code{or} from \link{Chapter 1}:
	\begin{itemize}

		\item
			\code{and}:
			The expressions are evaluated from left to right.
			If any expression evaluates to false, false is returned;
			any remaining expressions are not evaluated.
			If all the expressions evaluate to true values, the value of the last expression is returned.
			If there are no expressions then true is returned.


		\item
			\code{or}:
			The expressions are evaluated from left to right.
			If any expression evaluates to a true value, that value is returned;
			any remaining expressions are not evaluated.
			If all expressions evaluate to false, or if there are no expressions, then false is returned.

	\end{itemize}

	Install \code{and} and \code{or} as new special forms for the evaluator by defining appropriate syntax procedures and evaluation procedures \code{eval-and} and \code{eval-or}.
	Alternatively, show how to implement \code{and} and \code{or} as derived expressions.
\end{exercise}



\begin{exercise}
	\label{Exercise 4.5}
	Scheme allows an additional syntax for \code{cond} clauses, \code{(⟨\var{test}⟩ => ⟨\var{recipient}⟩)}.
	If \code{⟨\var{test}⟩} evaluates to a true value, then \code{⟨\var{recipient}⟩} is evaluated.
	Its value must be a procedure of one argument;
	this procedure is then invoked on the value of the \code{⟨\var{test}⟩}, and the result is returned as the value of the \code{cond} expression.
	For example
	\begin{scheme}
	  (cond ((assoc 'b '((a 1) (b 2))) => cadr)
	        (else false))
	\end{scheme}
	returns \( 2 \).
	Modify the handling of \code{cond} so that it supports this extended syntax.
\end{exercise}



\begin{exercise}
	\label{Exercise 4.6}
	\code{let} expressions are derived expressions, because
	\begin{scheme}
	  (let ((⟨~\var{var}\ind{1}~⟩ ⟨~\var{exp}\ind{1}⟩~) … (⟨~\var{var}\ind{n}~⟩ ⟨~\var{exp}\ind{n}~⟩))
	    ⟨~\var{body}~⟩)
	\end{scheme}
	is equivalent to
	\begin{scheme}
	  ((lambda (⟨~\var{var}\ind{1}~⟩ … ⟨~\var{var}\ind{n}~⟩)
	     ⟨~\var{body}~⟩)
	     ⟨~\var{exp}\ind{1}~⟩
	     ⋮
	     ⟨~\var{exp}\ind{n}~⟩
	\end{scheme}
	Implement a syntactic transformation \code{let->combination} that reduces evaluating \code{let} expressions to evaluating combinations of the type shown above, and add the appropriate clause to \code{eval} to handle \code{let}
	expressions.
\end{exercise}



\begin{exercise}
	\label{Exercise 4.7}
	\code{let*} is similar to \code{let}, except that the bindings of the \code{let*} variables are performed sequentially from left to right, and each binding is made in an environment in which all of the preceding bindings are visible.
	For example
	\begin{scheme}
	  (let* ((x 3)  (y (+ x 2))  (z (+ x y 5)))
	    (* x z))
	\end{scheme}
	returns \( 39 \).
	Explain how a \code{let*} expression can be rewritten as a set of nested \code{let} expressions, and write a procedure \code{let*->nested-lets} that performs this transformation.
	If we have already implemented \code{let} (\link{Exercise 4.6}) and we want to extend the evaluator to handle \code{let*}, is it sufficient to add a clause to \code{eval} whose action is
	\begin{scheme}
	  (eval (let*->nested-lets exp) env)
	\end{scheme}
	or must we explicitly expand \code{let*} in terms of non-derived expressions?
\end{exercise}



\begin{exercise}
	\label{Exercise 4.8}
	“Named \code{let}” is a variant of \code{let} that has the form
	\begin{scheme}
		(let ⟨~\var{var}~⟩ ⟨~\var{bindings}~⟩ ⟨~\var{body}~⟩)
	\end{scheme}
	The \code{⟨\var{bindings}⟩} and \code{⟨\var{body}⟩} are just as in ordinary \code{let}, except that \code{⟨\var{var}⟩} is bound within \code{⟨\var{body}⟩} to a procedure whose body is \code{⟨\var{body}⟩} and whose parameters are the variables in the \code{⟨\var{bindings}⟩}.
	Thus, one can repeatedly execute the \code{⟨\var{body}⟩} by invoking the procedure named \code{⟨\var{var}⟩}.
	For example, the iterative Fibonacci procedure (\link{Section 1.2.2}) can be rewritten using named \code{let} as follows:
	\begin{scheme}
	  (define (fib n)
	    (let fib-iter ((a 1)
	                   (b 0)
	                   (count n))
	      (if (= count 0)
	          b
	          (fib-iter (+ a b) a (- count 1)))))
	\end{scheme}
	Modify \code{let->combination} of \link{Exercise 4.6} to also support named \code{let}.
\end{exercise}



\begin{exercise}
	\label{Exercise 4.9}
	Many languages support a variety of iteration constructs, such as \code{do}, \code{for}, \code{while}, and \code{until}.
	In Scheme, iterative processes can be expressed in terms of ordinary procedure calls, so special iteration constructs provide no essential gain in computational power.
	On the other hand, such constructs are often convenient.
	Design some iteration constructs, give examples of their use, and show how to implement them as derived expressions.
\end{exercise}



\begin{exercise}
	\label{Exercise 4.10}
	By using data abstraction, we were able to write an \code{eval} procedure that is independent of the particular syntax of the language to be evaluated.
	To illustrate this, design and implement a new syntax for Scheme by modifying the procedures in this section, without changing \code{eval} or \code{apply}.
\end{exercise}
