\subsection{The Rules for Evaluation}
\label{Section 3.2.1}

The overall specification of how the interpreter evaluates a combination remains the same as when we first introduced it in \cref{Section 1.1.3}:
\begin{itemize}

	\item
		To evaluate a combination:
		\begin{enumerate}[leftmargin = *]

			\item
				Evaluate the subexpressions of the combination.%
				\footnote{
					Assignment introduces a subtlety into step 1 of the evaluation rule.
					As shown in \cref{Exercise 3.8}, the presence of assignment allows us to write expressions that will produce different values depending on the order in which the subexpressions in a combination are evaluated.
					Thus, to be precise, we should specify an evaluation order in step 1 (e.g., left to right or right to left).
					However, this order should always be considered to be an implementation detail, and one should never write programs that depend on some particular order.
					For instance, a sophisticated compiler might optimize a program by varying the order in which subexpressions are evaluated.
				}

			\item
				Apply the value of the operator subexpression to the values of the operand subexpressions.

		\end{enumerate}
\end{itemize}
The environment model of evaluation replaces the substitution model in specifying what it means to apply a compound procedure to arguments.

In the environment model of evaluation, a procedure is always a pair consisting of some code and a pointer to an environment.
Procedures are created in one way only:
by evaluating a λ-expression.
This produces a procedure whose code is obtained from the text of the λ-expression and whose environment is the environment in which the λ-expression was evaluated to produce the procedure.
For example, consider the procedure definition
\begin{scheme}
  (define (square x)
    (* x x))
\end{scheme}
evaluated in the global environment.
The procedure definition syntax is just syntactic sugar for an underlying implicit λ-expression.
It would have been equivalent to have used
\begin{scheme}
  (define square
    (lambda (x) (* x x)))
\end{scheme}
which evaluates \code{(lambda (x) (* x x))} and binds \code{square} to the resulting value, all in the global environment.

\begin{figure}[tb]
	\centering
	\includesvg[width=49mm]{fig/chap3/Fig3.2b.svg}
	\caption{
		Environment structure produced by evaluating \code{(define (square x) (* x x))} in the global environment.
	}
	\label{Figure 3.2}
\end{figure}

\cref{Figure 3.2} shows the result of evaluating this \code{define} expression.
The procedure object is a pair whose code specifies that the procedure has one formal parameter, namely \code{x}, and a procedure body \code{(* x x)}.
The environment part of the procedure is a pointer to the global environment, since that is the environment in which the λ-expression was evaluated to produce the procedure.
A new binding, which associates the procedure object with the symbol \code{square}, has been added to the global frame.
In general, \code{define} creates definitions by adding bindings to frames.

Now that we have seen how procedures are created, we can describe how procedures are applied.
The environment model specifies:
To apply a procedure to arguments, create a new environment containing a frame that binds the parameters to the values of the arguments.
The enclosing environment of this frame is the environment specified by the procedure.
Now, within this new environment, evaluate the procedure body.

\begin{figure}[tb]
	\centering
	\includesvg[width=78mm]{fig/chap3/Fig3.3b.svg}
	\caption{
		Environment created by evaluating \code{(square 5)} in the global environment.
	}
	\label{Figure 3.3}
\end{figure}

To show how this rule is followed, \cref{Figure 3.3} illustrates the environment structure created by evaluating the expression \code{(square 5)} in the global environment, where \code{square} is the procedure generated in \cref{Figure 3.2}.
Applying the procedure results in the creation of a new environment, labeled E1 in the figure, that begins with a frame in which \code{x}, the formal parameter for the procedure, is bound to the argument \( 5 \).
The pointer leading upward from this frame shows that the frame’s enclosing environment is the global environment.
The global environment is chosen here, because this is the environment that is indicated as part of the \code{square} procedure object.
Within E1, we evaluate the body of the procedure, \code{(* x x)}.
Since the value of \code{x} in E1 is \( 5 \), the result is \code{(* 5 5)}, or \( 25 \).

The environment model of procedure application can be summarized by two rules:
\begin{itemize}

	\item
		A procedure object is applied to a set of arguments by constructing a frame, binding the formal parameters of the procedure to the arguments of the call, and then evaluating the body of the procedure in the context of the new environment constructed.
		The new frame has as its enclosing environment the environment part of the procedure object being applied.

	\item
		A procedure is created by evaluating a λ-expression relative to a given environment.
		The resulting procedure object is a pair consisting of the text of the λ-expression and a pointer to the environment in which the procedure was created.

\end{itemize}

We also specify that defining a symbol using \code{define} creates a binding in the current environment frame and assigns to the symbol the indicated value.%
\footnote{
	If there is already a binding for the variable in the current frame, then the binding is changed.
	This is convenient because it allows redefinition of symbols;
	however, it also means that \code{define} can be used to change values, and this brings up the issues of assignment without explicitly using \code{set!}.
	Because of this, some people prefer redefinitions of existing symbols to signal errors or warnings.
}
Finally, we specify the behavior of \code{set!}, the operation that forced us to introduce the environment model in the first place.
Evaluating the expression \code{(set! ⟨\var{variable}⟩ ⟨\var{value}⟩} in some environment locates the binding of the variable in the environment and changes that binding to indicate the new value.
That is, one finds the first frame in the environment that contains a binding for the variable and modifies that frame.
If the variable is unbound in the environment, then \code{set!} signals an error.

These evaluation rules, though considerably more complex than the substitution model, are still reasonably straightforward.
Moreover, the evaluation model, though abstract, provides a correct description of how the interpreter evaluates expressions.
In \cref{Chapter 4} we shall see how this model can serve as a blueprint for implementing a working interpreter.
The following sections elaborate the details of the model by analyzing some illustrative programs.
