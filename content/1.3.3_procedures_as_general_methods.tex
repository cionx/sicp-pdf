\subsection{Procedures as General Methods}
\label{Section 1.3.3}

We introduced compound procedures in \cref{Section 1.1.4} as a mechanism for abstracting patterns of numerical operations so as to make them independent of the particular numbers involved.
With higher-order procedures, such as the \code{integral} procedure of \cref{Section 1.3.1}, we began to see a more powerful kind of abstraction:
procedures used to express general methods of computation, independent of the particular functions involved.
In this section we discuss two more elaborate examples---general methods for finding zeros and fixed points of functions---and show how these methods can be expressed directly as procedures.



\subsubsection*{Finding roots of equations by the half-interval method}

The \newterm{half-interval method} is a simple but powerful technique for finding roots of an equation \( f(x) = 0 \), where \( f \) is a continuous function.
The idea is that, if we are given points \( a \) and \( b \) such that \( f(a) < 0 < f(b) \), then \( f \) must have at least one zero between \( a \) and \( b \).
To locate a zero, let \( x \) be the average of \( a \) and \( b \), and compute \( f(x) \).
If \( f(x) > 0 \), then \( f \) must have a zero between \( a \) and \( x \).
If \( f(x) < 0 \), then \( f \) must have a zero between \( x \) and \( b \).
Continuing in this way, we can identify smaller and smaller intervals on which \( f \) must have a zero.
When we reach a point where the interval is small enough, the process stops.
Since the interval of uncertainty is reduced by half at each step of the process, the number of steps required grows as \( Θ(\log(L / T)) \), where \( L \) is the length of the original interval and \( T \) is the error tolerance (that is, the size of the interval we will consider “small enough”).
Here is a procedure that implements this strategy:
\begin{scheme}
  (define (search f neg-point pos-point)
    (let ((midpoint (average neg-point pos-point)))
      (if (close-enough? neg-point pos-point)
          midpoint
          (let ((test-value (f midpoint)))
            (cond ((positive? test-value)
                   (search f neg-point midpoint))
                  ((negative? test-value)
                   (search f midpoint pos-point))
                  (else midpoint))))))
\end{scheme}

We assume that we are initially given the function \( f \) together with points at which its values are negative and positive.
We first compute the midpoint of the two given points.
Next we check to see if the given interval is small enough, and if so we simply return the midpoint as our answer.
Otherwise, we compute as a test value the value of \( f \) at the midpoint.
If the test value is positive, then we continue the process with a new interval running from the original negative point to the midpoint.
If the test value is negative, we continue with the interval from the midpoint to the positive point.
Finally, there is the possibility that the test value is \( 0 \), in which case the midpoint is itself the root we are searching for.

To test whether the endpoints are “close enough” we can use a procedure similar to the one used in \cref{Section 1.1.7} for computing square roots:%
\footnote{
	We have used \( 0.001 \) as a representative “small” number to indicate a tolerance for the acceptable error in a calculation.
	The appropriate tolerance for a real calculation depends upon the problem to be solved and the limitations of the computer and the algorithm.
	This is often a very subtle consideration, requiring help from a numerical analyst or some other kind of magician.
}
\begin{scheme}
  (define (close-enough? x y)
    (< (abs (- x y)) 0.001))
\end{scheme}

\code{search} is awkward to use directly, because we can accidentally give it points at which \( f \)’s values do not have the required sign, in which case we get a wrong answer.
Instead we will use \code{search} via the following procedure, which checks to see which of the endpoints has a negative function value and which has a positive value, and calls the \code{search} procedure accordingly.
If the function has the same sign on the two given points, the half-interval method cannot be used, in which case the procedure signals an error.%
\footnote{
	This can be accomplished using \code{error}, which takes as arguments a number of items that are printed as error messages.
}
\begin{scheme}
  (define (half-interval-method f a b)
    (let ((a-value (f a))
          (b-value (f b)))
      (cond ((and (negative? a-value) (positive? b-value))
             (search f a b))
            ((and (negative? b-value) (positive? a-value))
             (search f b a))
            (else
             (error "Values are not of opposite sign" a b)))))
\end{scheme}

The following example uses the half-interval method to approximate \( π \) as the root between \( 2 \) and \( 4 \) of \( \sin x = 0 \):
\begin{scheme}
  (half-interval-method sin 2.0 4.0)
  ~\outprint{3.14111328125}~
\end{scheme}

Here is another example, using the half-interval method to search for a root of the equation \( x^3 - 2x - 3 = 0 \) between \( 1 \) and \( 2 \):
\begin{scheme}
  (half-interval-method (lambda (x) (- (* x x x) (* 2 x) 3))
                        1.0
                        2.0)
  ~\outprint{1.89306640625}~
\end{scheme}



\subsubsection*{Finding fixed points of functions}

A number \( x \) is called a \newterm{fixed point} of a function \( f \) if \( x \) satisfies the equation \( f(x) = x \).
For some functions \( f \) we can locate a fixed point by beginning with an initial guess and applying \( f \) repeatedly,
\[
	f(x)       \,, \quad
	f(f(x))    \,, \quad
	f(f(f(x))) \,, \quad
	\dotsc
\]
until the value does not change very much.
Using this idea, we can devise a procedure \code{fixed-point} that takes as inputs a function and an initial guess and produces an approximation to a fixed point of the function.
We apply the function repeatedly until we find two successive values whose difference is less than some prescribed tolerance:
\begin{scheme}
  (define tolerance 0.00001)

  (define (fixed-point f first-guess)
    (define (close-enough? v1 v2)
      (< (abs (- v1 v2))
         tolerance))
    (define (try guess)
      (let ((next (f guess)))
        (if (close-enough? guess next)
            next
            (try next))))
    (try first-guess))
\end{scheme}
For example, we can use this method to approximate the fixed point of the cosine function, starting with \( 1 \) as an initial approximation:%
\footnote{
	Try this during a boring lecture:
	Set your calculator to radians mode and then repeatedly press the \keys{cos} button until you obtain the fixed point.
}
\begin{scheme}
  (fixed-point cos 1.0)
  ~\outprint{.7390822985224023}~
\end{scheme}
Similarly, we can find a solution to the equation \( y = \sin y + \cos y \):
\begin{scheme}
  (fixed-point (lambda (y) (+ (sin y) (cos y)))
               1.0)
  ~\outprint{1.2587315962971173}~
\end{scheme}

The fixed-point process is reminiscent of the process we used for finding square roots in \cref{Section 1.1.7}.
Both are based on the idea of repeatedly improving a guess until the result satisfies some criterion.
In fact, we can readily formulate the square-root computation as a fixed-point search.
Computing the square root of some number \( x \) requires finding a \( y \) such that \( y^2 = x \).
Putting this equation into the equivalent form \( y = x / y \), we recognize that we are looking for a fixed point of the function%
\footnote{
	\( \mapsto \) (pronounced “maps to”) is the mathematician’s way of writing \code{lambda}.
	\( y \mapsto x / y \) means \code{(lambda (y) (/ x y))}, that is, the function whose value at \( y \) is \( x / y \).
}
\( y \mapsto x / y \), and we can therefore try to compute square roots as
\begin{scheme}
  (define (sqrt x)
    (fixed-point (lambda (y) (/ x y))
                 1.0))
\end{scheme}

Unfortunately, this fixed-point search does not converge.
Consider an initial guess \( y_1 \).
The next guess is \( y_2 = x / y_1 \) and the next guess is \( y_3 = x / y_2 = x / (x / y_1) = y_1 \).
This results in an infinite loop in which the two guesses \( y_1 \) and \( y_2 \) repeat over and over, oscillating about the answer.

One way to control such oscillations is to prevent the guesses from changing so much.
Since the answer is always between our guess \( y \) and \( x / y \), we can make a new guess that is not as far from \( y \) as \( x / y \) by averaging \( y \) with \( x / y \), so that the next guess after \( y \) is \( \frac{1}{2} (y + x / y) \) instead of \( x / y \).
The process of making such a sequence of guesses is simply the process of looking for a fixed point of \( y \mapsto \frac{1}{2} (y + x / y) \):
\begin{scheme}
  (define (sqrt x)
    (fixed-point (lambda (y) (average y (/ x y)))
                 1.0))
\end{scheme}
(Note that \( y = \frac{1}{2} (y + x / y) \) is a simple transformation of the equation \( y = x / y; \) to derive it, add \( y \) to both sides of the equation and divide by \( 2 \).)

With this modification, the square-root procedure works.
In fact, if we unravel the definitions, we can see that the sequence of approximations to the square root generated here is precisely the same as the one generated by our original square-root procedure of \cref{Section 1.1.7}.
This approach of averaging successive approximations to a solution, a technique that we call \newterm{average damping}, often aids the convergence of fixed-point searches.



\begin{exercise}
	\label{Exercise 1.35}
	Show that the golden ratio \( ϕ \) (\cref{Section 1.2.2}) is a fixed point of the transformation \( x \mapsto 1 + 1 / x \), and use this fact to compute \( ϕ \) by means of the \code{fixed-point} procedure.
\end{exercise}



\begin{exercise}
	\label{Exercise 1.36}
	Modify \code{fixed-point} so that it prints the sequence of approximations it generates, using the \code{newline} and \code{display} primitives shown in \cref{Exercise 1.22}.
	Then find a solution to \( x^x = 1000 \) by finding a fixed point of \( x \mapsto \log(1000) / \log(x) \).
	(Use Scheme’s primitive \code{log} procedure, which computes natural logarithms.)
	Compare the number of steps this takes with and without average damping.
	(Note that you cannot start \code{fixed-point} with a guess of \( 1 \), as this would cause division by \( \log(1) = 0 \).)
\end{exercise}



\begin{exercise}
	\label{Exercise 1.37}
	\begin{enumerate}[label = \alph*., leftmargin = *]

		\item
			An infinite \newterm{continued fraction} is an expression of the form
			\[
				f = \cfrac{N_1}{D_1 + \cfrac{N_2}{D_2 + \cfrac{N_3}{D_3 + \dotsb}}} \,.
			\]
			As an example, one can show that the infinite continued fraction expansion with the \( N_i \) and the \( D_i \) all equal to \( 1 \) produces \( 1 / ϕ \), where \( ϕ \) is the golden ratio (described in \cref{Section 1.2.2}).
			One way to approximate an infinite continued fraction is to truncate the expansion after a given number of terms.
			Such a truncation---a so-called \newterm{\( k \)-term finite continued fraction}---has the form
			\[
				\cfrac{N_1}{D_1 + \cfrac{N_2}{\ddots + \cfrac{N_k}{D_k}}} \,.
			\]
			Suppose that \code{n} and \code{d} are procedures of one argument (the term index \( i \)) that return the \( N_i \) and \( D_i \) of the terms of the continued fraction.
			Define a procedure \code{cont-frac} such that evaluating \code{(cont-frac n d k)} computes the value of the \( k \)-term finite continued fraction.
			Check your procedure by approximating \( 1 / ϕ \) using
			\begin{scheme}
			  (cont-frac (lambda (i) 1.0)
			             (lambda (i) 1.0)
			             k)
			\end{scheme}
			for successive values of \code{k}.
			How large must you make \code{k} in order to get an approximation that is accurate to \( 4 \) decimal places?

		\item
			If your \code{cont-frac} procedure generates a recursive process, write one that generates an iterative process.
			If it generates an iterative process, write one that generates a recursive process.

	\end{enumerate}
\end{exercise}



\begin{exercise}
	\label{Exercise 1.38}
	In 1737, the Swiss mathematician Leonhard Euler published a memoir \booktitle{De Fractionibus Continuis}, which included a continued fraction expansion for \( e - 2 \), where \( e \) is the base of the natural logarithms.
	In this fraction, the \( N_i \) are all \( 1 \), and the \( D_i \) are successively \( 1, 2, 1, 1, 4, 1, 1, 6, 1, 1, 8, \dotsc \)
	Write a program that uses your \code{cont-frac} procedure from \cref{Exercise 1.37} to approximate \( e \), based on Euler’s expansion.
\end{exercise}



\begin{exercise}
	\label{Exercise 1.39}
	A continued fraction representation of the tangent function was published in 1770 by the German mathematician J.H. Lambert:
	\[
		\tan x = \cfrac{x}{1 - \cfrac{x^2}{3 - \cfrac{x^2}{5 - {⋱}}}}\,,
	\]
	where \( x \) is in radians.
	Define a procedure \code{(tan-cf x k)} that computes an approximation to the tangent function based on Lambert’s formula.
	\code{k} specifies the number of terms to compute, as in \cref{Exercise 1.37}.
\end{exercise}
