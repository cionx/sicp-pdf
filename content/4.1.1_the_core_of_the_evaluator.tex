\subsection{The Core of the Evaluator}
\label{Section 4.1.1}

The evaluation process can be described as the interplay between two procedures:
\code{eval} and \code{apply}.



\subsubsection*{Eval}

\code{eval} takes as arguments an expression and an environment.
It classifies the expression and directs its evaluation.
\code{eval} is structured as a case analysis of the syntactic type of the expression to be evaluated.
In order to keep the procedure general, we express the determination of the type of an expression abstractly, making no commitment to any particular representation for the various types of expressions.
Each type of expression has a predicate that tests for it and an abstract means for selecting its parts.
This \newterm{abstract syntax} makes it easy to see how we can change the syntax of the language by using the same evaluator, but with a different collection of syntax procedures.

\begin{itemize}[leftmargin=*]
	\item
		\textit{Primitive expressions}
		\begin{itemize}

			\item
				For self-evaluating expressions, such as numbers, \code{eval} returns the expression itself.

			\item
				\code{eval} must look up variables in the environment to find their values.

		\end{itemize}

	\item
		\textit{Special forms}
		\begin{itemize}

			\item
				For quoted expressions, \code{eval} returns the expression that was quoted.

			\item
				An assignment to (or a definition of) a variable must recursively call \code{eval} to compute the new value to be associated with the variable.
				The environment must be modified to change (or create) the binding of the variable.

			\item
				An \code{if} expression requires special processing of its parts, so as to evaluate the consequent if the predicate is true, and otherwise to evaluate the alternative.

			\item
				A \code{lambda} expression must be transformed into an applicable procedure by packaging together the parameters and body specified by the \code{lambda} expression with the environment of the evaluation.

			\item
				A \code{begin} expression requires evaluating its sequence of expressions in the order in which they appear.

			\item
				A case analysis (\code{cond}) is transformed into a nest of \code{if} expressions and then evaluated.

		\end{itemize}

	\item
		\textit{Combinations}
		\begin{itemize}

			\item
				For a procedure application, \code{eval} must recursively evaluate the operator part and the operands of the combination.
				The resulting procedure and arguments are passed to \code{apply}, which handles the actual procedure application.

		\end{itemize}
\end{itemize}
%
Here is the definition of \code{eval}:
\begin{scheme}
  (define (eval exp env)
    (cond ((self-evaluating? exp) exp)
          ((variable? exp) (lookup-variable-value exp env))
          ((quoted? exp) (text-of-quotation exp))
          ((assignment? exp) (eval-assignment exp env))
          ((definition? exp) (eval-definition exp env))
          ((if? exp) (eval-if exp env))
          ((lambda? exp) (make-procedure (lambda-parameters exp)
                                         (lambda-body exp)
                                         env))
          ((begin? exp)
           (eval-sequence (begin-actions exp) env))
          ((cond? exp) (eval (cond->if exp) env))
          ((application? exp)
           (apply (eval (operator exp) env)
                  (list-of-values (operands exp) env)))
          (else
           (error "Unknown expression type: EVAL" exp))))
\end{scheme}

For clarity, \code{eval} has been implemented as a case analysis using \code{cond}.
The disadvantage of this is that our procedure handles only a few distinguishable types of expressions, and no new ones can be defined without editing the definition of \code{eval}.
In most Lisp implementations, dispatching on the type of an expression is done in a data-directed style.
This allows a user to add new types of expressions that \code{eval} can distinguish, without modifying the definition of \code{eval} itself.
(See \cref{Exercise 4.3}.)



\subsubsection*{Apply}

\code{apply} takes two arguments, a procedure and a list of arguments to which the procedure should be applied.
\code{apply} classifies procedures into two kinds:
It calls \code{apply-primitive-procedure} to apply primitives;
it applies compound procedures by sequentially evaluating the expressions that make up the body of the procedure.
The environment for the evaluation of the body of a compound procedure is constructed by extending the base environment carried by the procedure to include a frame that binds the parameters of the procedure to the arguments to which the procedure is to be applied.
Here is the definition of \code{apply}:
\begin{scheme}
  (define (apply procedure arguments)
    (cond ((primitive-procedure? procedure)
           (apply-primitive-procedure procedure arguments))
          ((compound-procedure? procedure)
           (eval-sequence
             (procedure-body procedure)
             (extend-environment
               (procedure-parameters procedure)
               arguments
               (procedure-environment procedure))))
          (else
           (error
            "Unknown procedure type: APPLY" procedure))))
\end{scheme}



\subsubsection*{Procedure arguments}

When \code{eval} processes a procedure application, it uses \code{list-of-values} to produce the list of arguments to which the procedure is to be applied.
\code{list-of-values} takes as an argument the operands of the combination.
It evaluates each operand and returns a list of the corresponding values:%
\footnote{
	We could have simplified the \code{application?} clause in \code{eval} by using \code{map} (and stipulating that \code{operands} returns a list) rather than writing an explicit \code{list-of-values} procedure.
	We chose not to use \code{map} here to emphasize the fact that the evaluator can be implemented without any use of higher-order procedures (and thus could be written in a language that doesn’t have higher-order procedures), even though the language that it supports will include higher-order procedures.
}
\begin{scheme}
  (define (list-of-values exps env)
    (if (no-operands? exps)
        '()
        (cons (eval (first-operand exps) env)
              (list-of-values (rest-operands exps) env))))
\end{scheme}



\subsubsection*{Conditionals}

\code{eval-if} evaluates the predicate part of an \code{if} expression in the given environment.
If the result is true, \code{eval-if} evaluates the consequent, otherwise it evaluates the alternative:
\begin{scheme}
  (define (eval-if exp env)
    (if (true? (eval (if-predicate exp) env))
        (eval (if-consequent exp) env)
        (eval (if-alternative exp) env)))
\end{scheme}

The use of \code{true?} in \code{eval-if} highlights the issue of the connection between an implemented language and an implementation language.
The \code{if-predicate} is evaluated in the language being implemented and thus yields a value in that language.
The interpreter predicate \code{true?} translates that value into a value that can be tested by the \code{if} in the implementation language:
The metacircular representation of truth might not be the same as that of the underlying Scheme.%
\footnote{
	In this case, the language being implemented and the implementation language are the same.
	Contemplation of the meaning of \code{true?} here yields expansion of consciousness without the abuse of substance.
}



\subsubsection*{Sequences}

\code{eval-sequence} is used by \code{apply} to evaluate the sequence of expressions in a procedure body and by \code{eval} to evaluate the sequence of expressions in a \code{begin} expression.
It takes as arguments a sequence of expressions and an environment, and evaluates the expressions in the order in which they occur.
The value returned is the value of the final expression.
\begin{scheme}
  (define (eval-sequence exps env)
    (cond ((last-exp? exps)
           (eval (first-exp exps) env))
          (else
           (eval (first-exp exps) env)
           (eval-sequence (rest-exps exps) env))))
\end{scheme}



\subsubsection*{Assignments and definitions}

The following procedure handles assignments to variables.
It calls \code{eval} to find the value to be assigned and transmits the variable and the resulting value to \code{set-variable-value!} to be installed in the designated environment.
\begin{scheme}
  (define (eval-assignment exp env)
    (set-variable-value! (assignment-variable exp)
                         (eval (assignment-value exp) env)
                         env)
    'ok)
\end{scheme}
Definitions of variables are handled in a similar manner.%
\footnote{
	This implementation of \code{define} ignores a subtle issue in the handling of internal definitions, although it works correctly in most cases.
	We will see what the problem is and how to solve it in \cref{Section 4.1.6}.
}
\begin{scheme}
  (define (eval-definition exp env)
    (define-variable! (definition-variable exp)
                      (eval (definition-value exp) env)
                      env)
    'ok)
\end{scheme}
We have chosen here to return the symbol \code{ok} as the value of an
assignment or a definition.%
\footnote{
	As we said when we introduced \code{define} and \code{set!}, these values are implementation-dependent in Scheme---that is, the implementor can choose what value to return.
}



\begin{exercise}
\label{Exercise 4.1}
	Notice that we cannot tell whether the metacircular evaluator evaluates operands from left to right or from right to left.
	Its evaluation order is inherited from the underlying Lisp:
	If the arguments to \code{cons} in \code{list-of-values} are evaluated from left to right, then \code{list-of-values} will evaluate operands from left to right;
	and if the arguments to \code{cons} are evaluated from right to left, then \code{list-of-values} will evaluate operands from right to left.

	Write a version of \code{list-of-values} that evaluates operands from left to right regardless of the order of evaluation in the underlying Lisp.
	Also write a version of \code{list-of-values} that evaluates operands from right to left.
\end{exercise}
