\subsection{Exponentiation}
\label{Section 1.2.4}

Consider the problem of computing the exponential of a given number.
We would like a procedure that takes as arguments a base \( b \) and a positive integer exponent \( n \) and computes \( b^n \).
One way to do this is via the recursive definition
\begin{align*}
	b^n &= b ⋅ b^{n-1} \,, \\
	b^0 &= 1 \,,
\end{align*}
which translates readily into the procedure
\begin{scheme}
  (define (expt b n)
    (if (= n 0)
        1
        (* b (expt b (- n 1)))))
\end{scheme}
This is a linear recursive process, which requires \( Θ(n) \) steps and \( Θ(n) \) space.
Just as with factorial, we can readily formulate an equivalent linear iteration:
\begin{scheme}
  (define (expt b n)
    (expt-iter b n 1))

  (define (expt-iter b counter product)
    (if (= counter 0)
        product
        (expt-iter b
                   (- counter 1)
                   (* b product))))
\end{scheme}
This version requires \( Θ(n) \) steps and \( Θ(1) \) space.

We can compute exponentials in fewer steps by using successive squaring.
For instance, rather than computing \( b^8 \) as
\[
   b ⋅ (b ⋅ (b ⋅ (b ⋅ (b ⋅ (b ⋅ (b ⋅ b)))))) \,,
\]
we can compute it using three multiplications:
\begin{align*}
	b^2 &= b ⋅ b, \\
	b^4 &= b^2 ⋅ b^2, \\
	b^8 &= b^4 ⋅ b^4.
\end{align*}

This method works fine for exponents that are powers of \( 2 \).
We can also take advantage of successive squaring in computing exponentials in general if we use the rule
\[
	b^n
	=
	\begin{cases*}
		(b^{n / 2})^2 & if \( n \) is even, \\
		b ⋅ b^{n - 1} & if \( n \) is odd.
	\end{cases*}
\]
We can express this method as a procedure:
\begin{scheme}
  (define (fast-expt b n)
    (cond ((= n 0) 1)
          ((even? n) (square (fast-expt b (/ n 2))))
          (else (* b (fast-expt b (- n 1))))))
\end{scheme}
where the predicate to test whether an integer is even is defined in terms of the primitive procedure \code{remainder} by
\begin{scheme}
  (define (even? n)
    (= (remainder n 2) 0))
\end{scheme}
The process evolved by \code{fast-expt} grows logarithmically with \( n \) in both space and number of steps.
To see this, observe that computing \( b^{2 n} \) using \code{fast-expt} requires only one more multiplication than computing \( b^n \).
The size of the exponent we can compute therefore doubles (approximately) with every new multiplication we are allowed.
Thus, the number of multiplications required for an exponent of \( n \) grows about as fast as \( \log_2 n \).
The process has \( Θ(\log n) \) growth.%
\footnote{
	More precisely, the number of multiplications required is equal to \( 1 \) less than \( \log_2 n \) plus the number of ones in the binary representation of \( n \).
	This total is always less than \( 2 \log_2 n \).
	The arbitrary constants \( k_1 \) and \( k_2 \) in the definition of order notation imply that, for a logarithmic process, the base to which logarithms are taken does not matter, so all such processes are described as \( Θ(\log n) \).
}

The difference between \( Θ(\log n) \) growth and \( Θ(n) \) growth becomes striking as \( n \) becomes large.
For example, \code{fast-expt} for \( n = 1000 \) requires only 14 multiplications.%
\footnote{
	You may wonder why anyone would care about raising numbers to the 1000th power.
	See \cref{Section 1.2.6}.
}
It is also possible to use the idea of successive squaring to devise an iterative algorithm that computes exponentials with a logarithmic number of steps (see \cref{Exercise 1.16}), although, as is often the case with iterative algorithms, this is not written down so straightforwardly as the recursive algorithm.%
\footnote{
	This iterative algorithm is ancient.
	It appears in the \booktitle{Chandah-sutra} by Áchárya Pingala, written before 200~\acronym{B.C.}
	See \cref{Knuth 1981}, Section 4.6.3, for a full discussion and analysis of this and other methods of exponentiation.
}



\begin{exercise}
	\label{Exercise 1.16}
	Design a procedure that evolves an iterative exponentiation process that uses successive squaring and uses a logarithmic number of steps, as does \code{fast-expt}.
	(Hint:
	Using the observation that \( (b^{n / 2})^2 = (b^2)^{n / 2} \), keep, along with the exponent \( n \) and the base \( b \), an additional state variable \( a \), and define the state transformation in such a way that the product \( a b^n \) is unchanged from state to state.
	At the beginning of the process \( a \) is taken to be \( 1 \), and the answer is given by the value of \( a \) at the end of the process.
	In general, the technique of defining an \newterm{invariant quantity} that remains unchanged from state to state is a powerful way to think about the design of iterative algorithms.)
\end{exercise}



\begin{exercise}
	\label{Exercise 1.17}
	The exponentiation algorithms in this section are based on performing exponentiation by means of repeated multiplication.
	In a similar way, one can perform integer multiplication by means of repeated addition.
	The following multiplication procedure (in which it is assumed that our language can only add, not multiply) is analogous to the \code{expt} procedure:
	\begin{scheme}
	  (define (* a b)
	    (if (= b 0)
	        0
	        (+ a (* a (- b 1)))))
	\end{scheme}
	This algorithm takes a number of steps that is linear in \code{b}.
	Now suppose we include, together with addition, operations \code{double}, which doubles an integer, and \code{halve}, which divides an (even) integer by \( 2 \).
	Using these, design a multiplication procedure analogous to \code{fast-expt} that uses a logarithmic number of steps.
\end{exercise}



\begin{exercise}
	\label{Exercise 1.18}
	Using the results of \cref{Exercise 1.16} and \cref{Exercise 1.17}, devise a procedure that generates an iterative process for multiplying two integers in terms of adding, doubling, and halving and uses a logarithmic number of steps.%
	\footnote{
		This algorithm, which is sometimes known as the “Russian peasant method” of multiplication, is ancient.
		Examples of its use are found in the Rhind Papyrus, one of the two oldest mathematical documents in existence, written about 1700~\acronym{B.C.} (and copied from an even older document) by an Egyptian scribe named A’h-mose.
	}
\end{exercise}



\begin{exercise}
	\label{Exercise 1.19}
	There is a clever algorithm for computing the Fibonacci numbers in a logarithmic number of steps.
	Recall the transformation of the state variables \( a \) and \( b \) in the \code{fib-iter} process of \cref{Section 1.2.2}:
	\( a \gets a + b \) and \( b \gets a \).
	Call this transformation \( T \), and observe that applying \( T \) over and over again \( n \) times, starting with \( 1 \) and \( 0 \), produces the pair \( \Fib (n + 1) \) and \( \Fib(n) \).
	In other words, the Fibonacci numbers are produced by applying \( T^n \), the \( n \)\nth{th} power of the transformation \( T \), starting with the pair \( (1, 0) \).
	Now consider \( T \) to be the special case of \( p = 0 \) and \( q = 1 \) in a family of transformations \( T_{pq} \), where \( T_{pq} \) transforms the pair \( (a, b) \) according to \( a \gets bq + aq + ap \) and \( b \gets bp + aq \).
	Show that if we apply such a transformation \( T_{pq} \) twice, the effect is the same as using a single transformation \( T_{p' q'} \) of the same form, and compute \( p'\) and \( q' \) in terms of \( p \) and \( q \).
	This gives us an explicit way to square these transformations, and thus we can compute \( T^n \) using successive squaring, as in the \code{fast-expt} procedure.
	Put this all together to complete the following procedure, which runs in a logarithmic number of steps:%
	\footnote{
		This exercise was suggested to us by Joe Stoy, based on an example in \cref{Kaldewaij 1990}.
	}
	\begin{scheme}
	  (define (fib n)
	    (fib-iter 1 0 0 1 n))

	  (define (fib-iter a b p q count)
	    (cond ((= count 0) b)
	          ((even? count)
	           (fib-iter a
	                     b
	                     ~\cvar{⟨??⟩}~   ; ~\textrm{compute \( p' \)}~
	                     ~\cvar{⟨??⟩}~   ; ~\textrm{compute \( q' \)}~
	                     (/ count 2)))
	          (else (fib-iter (+ (* b q) (* a q) (* a p))
	                          (+ (* b p) (* a q))
	                          p
	                          q
	                          (- count 1)))))
	\end{scheme}
\end{exercise}
