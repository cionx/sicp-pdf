\subsection{Streams and Delayed Evaluation}
\label{Section 3.5.4}

The \code{integral} procedure at the end of the preceding section shows how we can use streams to model signal-processing systems that contain feedback loops.
The feedback loop for the adder shown in \cref{Figure 3.32} is modeled by the fact that \code{integral}’s internal stream \code{int} is defined in terms of itself:
\begin{scheme}
  (define int
    (cons-stream
     initial-value
     (add-streams (scale-stream integrand dt)
                  int)))
\end{scheme}
The interpreter’s ability to deal with such an implicit definition depends on the \code{delay} that is incorporated into \code{cons-stream}.
Without this \code{delay}, the interpreter could not construct \code{int} before evaluating both arguments to \code{cons-stream}, which would require that \code{int} already be defined.
In general, \code{delay} is crucial for using streams to model signal-processing systems that contain loops.
Without \code{delay}, our models would have to be formulated so that the inputs to any signal-processing component would be fully evaluated before the output could be produced.
This would outlaw loops.

\begin{figure}[tb]
	\centering
	\includesvg[width=67mm]{fig/chap3/Fig3.34.svg}
	\caption{
		An “analog computer circuit” that solves the equation \( dy / dt = f(y) \).
	}
	\label{Figure 3.34}
\end{figure}

Unfortunately, stream models of systems with loops may require uses of \code{delay} beyond the “hidden” \code{delay} supplied by \code{cons-stream}.
For instance, \cref{Figure 3.34} shows a signal-processing system for solving the differential equation \( dy / dt = f(y) \) where \( f \) is a given function.
The figure shows a mapping component, which applies \( f \) to its input signal, linked in a feedback loop to an integrator in a manner very similar to that of the analog computer circuits that are actually used to solve such equations.

Assuming we are given an initial value \( y_0 \) for \( y \), we could try to model this system using the procedure
\begin{scheme}
  (define (solve f y0 dt)
    (define y (integral dy y0 dt))
    (define dy (stream-map f y))
    y)
\end{scheme}
This procedure does not work, because in the first line of \code{solve} the call to \code{integral} requires that the input \code{dy} be defined, which does not happen until the second line of \code{solve}.

On the other hand, the intent of our definition does make sense, because we can, in principle, begin to generate the \code{y} stream without knowing \code{dy}.
Indeed, \code{integral} and many other stream operations have properties similar to those of \code{cons-stream}, in that we can generate part of the answer given only partial information about the arguments.
For \code{integral}, the first element of the output stream is the specified \code{initial-value}.
Thus, we can generate the first element of the output stream without evaluating the integrand \code{dy}.
Once we know the first element of \code{y}, the \code{stream-map} in the second line of \code{solve} can begin working to generate the first element of \code{dy}, which will produce the next element of \code{y}, and so on.

To take advantage of this idea, we will redefine \code{integral} to expect the integrand stream to be a \newterm{delayed argument}.
\code{integral} will \code{force} the integrand to be evaluated only when it is required to generate more than the first element of the output stream:
\begin{scheme}
  (define (integral delayed-integrand initial-value dt)
    (define int
      (cons-stream
       initial-value
       (let ((integrand (force delayed-integrand)))
         (add-streams (scale-stream integrand dt) int))))
    int)
\end{scheme}
Now we can implement our \code{solve} procedure by delaying the evaluation of \code{dy} in the definition of \code{y}:%
\footnote{
	This procedure is not guaranteed to work in all Scheme implementations, although for any implementation there is a simple variation that will work.
	The problem has to do with subtle differences in the ways that Scheme implementations handle internal definitions.
	(See \cref{Section 4.1.6}.)
}
\begin{scheme}
  (define (solve f y0 dt)
    (define y (integral (delay dy) y0 dt))
    (define dy (stream-map f y))
    y)
\end{scheme}
In general, every caller of \code{integral} must now \code{delay} the integrand argument.
We can demonstrate that the \code{solve} procedure works by approximating \( e ≈ 2.718 \) by computing the value at \( y = 1 \) of the solution to the differential equation \( dy / dt = y \) with initial condition \( y(0) = 1 \):
\begin{scheme}
  (stream-ref (solve (lambda (y) y)
                     1
                     0.001)
              1000)
  ~\outprint{2.716924}~
\end{scheme}



\begin{exercise}
	\label{Exercise 3.77}
	The \code{integral} procedure used above was analogous to the “implicit” definition of the infinite stream of integers in \cref{Section 3.5.2}.
	Alternatively, we can give a definition of \code{integral} that is more like \code{integers-starting-from} (also in \cref{Section 3.5.2}):
	\begin{smallscheme}
	  (define (integral integrand initial-value dt)
	    (cons-stream
	     initial-value
	     (if (stream-null? integrand)
	         the-empty-stream
	         (integral (stream-cdr integrand)
	                   (+ (* dt (stream-car integrand))
	                      initial-value)
	                   dt))))
	\end{smallscheme}
	When used in systems with loops, this procedure has the same problem as does our original version of \code{integral}.
	Modify the procedure so that it expects the \code{integrand} as a delayed argument and hence can be used in the \code{solve} procedure shown above.
\end{exercise}



\begin{figure}[tb]
	\centering
	\includesvg[width=91mm]{fig/chap3/Fig3.35a.svg}
	\caption{
		Signal-flow diagram for the solution to a second-order linear differential equation.
	}
	\label{Figure 3.35}
\end{figure}



\begin{exercise}
	\label{Exercise 3.78}
	Consider the problem of designing a signal-processing system to study the homogeneous second-order linear differential equation
	\[
		\frac{d^2 y}{d t^2} - a \frac{d y}{d t} - b y = 0 \,.
	\]
	The output stream, modeling \( y \), is generated by a network that contains a loop.
	This is because the value of \( d^2 y / d t^2 \) depends upon the values of \( y \) and \( dy / dt \) and both of these are determined by integrating \( d^2 y / dt^2 \).
	The diagram we would like to encode is shown in \cref{Figure 3.35}.
	Write a procedure \code{solve-2nd} that takes as arguments the constants \( a \), \( b \), and \( dt \) and the initial values \( y_0 \) and \( dy_0 \) for \( y \) and \( dy / dt \) and generates the stream of successive values of \( y \).
\end{exercise}



\begin{exercise}
	\label{Exercise 3.79}
	Generalize the \code{solve-2nd} procedure of \cref{Exercise 3.78} so that it can be used to solve general second-order differential equations \( d^2 y / d t^2 = f(dy / dt, y) \).
\end{exercise}



\begin{figure}[tb]
	\centering
	\includesvg[width=60mm]{fig/chap3/Fig3.36.svg}
	\caption{
		A series RLC circuit.
	}
	\label{Figure 3.36}
\end{figure}



\begin{exercise}
	\label{Exercise 3.80}
	A \newterm{series RLC circuit} consists of a resistor, a capacitor, and an inductor connected in series, as shown in \cref{Figure 3.36}.
	If \( R \), \( L \), and \( C \) are the resistance, inductance, and capacitance, then the relations between voltage (\( v \)) and current (\( i \)) for the three components are described by the equations
	\[
		v_R = i_R R \,, \quad
		v_L = L \frac{d i_L}{d t} \,, \quad
		i_C = C \frac{d v_C}{d t}
	\]
	and the circuit connections dictate the relations
	\[
		i_R = i_L = -i_C \,, \quad
		v_C = v_L + v_R \,.
	\]
	Combining these equations shows that the state of the circuit (summarized by \( v_C \), the voltage across the capacitor, and \( i_L \), the current in the inductor) is described by the pair of differential equations
	\[
		\frac{d v_C}{d t} = - \frac{i_L}{C} \,,
		\quad
		\frac{d L}{d t} = \frac{1}{L} v_C - \frac{R}{L} i_L \,.
		\]
	The signal-flow diagram representing this system of differential equations is shown in \cref{Figure 3.37}.
\end{exercise}



\begin{figure}[tb]
	\centering
	\includesvg[width=68mm]{fig/chap3/Fig3.37a.svg}
	\caption{
		A signal-flow diagram for the solution to a series RLC circuit.
	}
	\label{Figure 3.37}
\end{figure}



\begin{exercise}
	Write a procedure \code{RLC} that takes as arguments the parameters \( R \), \( L \), and \( C \) of the circuit and the time increment \( dt \).
	In a manner similar to that of the \code{RC} procedure of \cref{Exercise 3.73}, \code{RLC} should produce a procedure that takes the initial values of the state variables, \( v_{C_0} \) and \( i_{L_0} \), and produces a pair (using \code{cons}) of the streams of states \( v_C \) and \( i_L \).
	Using \code{RLC}, generate the pair of streams that models the behavior of a series RLC circuit with \( R \) = 1 ohm, \( C \) = 0.2 farad, \( L = 1 \) henry, \( dt \) = 0.1 second, and initial values \( i_{L_0} = 0 \) amps and \( v_{C_0} \) = 10 volts.
\end{exercise}



\subsubsection*{Normal-order evaluation}

The examples in this section illustrate how the explicit use of \code{delay} and \code{force} provides great programming flexibility, but the same examples also show how this can make our programs more complex.
Our new \code{integral} procedure, for instance, gives us the power to model systems with loops, but we must now remember that \code{integral} should be called with a delayed integrand, and every procedure that uses \code{integral} must be aware of this.
In effect, we have created two classes of procedures:
ordinary procedures and procedures that take delayed arguments.
In general, creating separate classes of procedures forces us to create separate classes of higher-order procedures as well.%
\footnote{
	This is a small reflection, in Lisp, of the difficulties that conventional strongly typed languages such as Pascal have in coping with higher-order procedures.
	In such languages, the programmer must specify the data types of the arguments and the result of each procedure:
	number, logical value, sequence, and so on.
	Consequently, we could not express an abstraction such as “map a given procedure \code{proc} over all the elements in a sequence” by a single higher-order procedure such as \code{stream-map}.
	Rather, we would need a different mapping procedure for each different combination of argument and result data types that might be specified for a \code{proc}.
	Maintaining a practical notion of “data type” in the presence of higher-order procedures raises many difficult issues.
	One way of dealing with this problem is illustrated by the language ML (\cref{Gordon et al. 1979}), whose “polymorphic data types” include templates for higher-order transformations between data types.
	Moreover, data types for most procedures in ML are never explicitly declared by the programmer.
	Instead, ML includes a \newterm{type-inferencing} mechanism that uses information in the environment to deduce the data types for newly defined procedures.
}

One way to avoid the need for two different classes of procedures is to make all procedures take delayed arguments.
We could adopt a model of evaluation in which all arguments to procedures are automatically delayed and arguments are forced only when they are actually needed (for example, when they are required by a primitive operation).
This would transform our language to use normal-order evaluation, which we first described when we introduced the substitution model for evaluation in \cref{Section 1.1.5}.
Converting to normal-order evaluation provides a uniform and elegant way to simplify the use of delayed evaluation, and this would be a natural strategy to adopt if we were concerned only with stream processing.
In \cref{Section 4.2}, after we have studied the evaluator, we will see how to transform our language in just this way.
Unfortunately, including delays in procedure calls wreaks havoc with our ability to design programs that depend on the order of events, such as programs that use assignment, mutate data, or perform input or output.
Even the single \code{delay} in \code{cons-stream} can cause great confusion, as illustrated by \cref{Exercise 3.51} and \cref{Exercise 3.52}.
As far as anyone knows, mutability and delayed evaluation do not mix well in programming languages, and devising ways to deal with both of these at once is an active area of research.
