\chapter*{Unofficial Texinfo Format}
\label{UTF}

This is the second edition \acronym{SICP} book, from Unofficial Texinfo Format.

You are probably reading it in an Info hypertext browser, such as the Info mode of Emacs.
You might alternatively be reading it \TeX{}-formatted on your screen or printer, though that would be silly.
And, if printed, expensive.
The freely-distributed official \acronym{HTML}-and-\acronym{GIF} format was first converted personally to Unofficial Texinfo Format (\acronym{UTF}) version~1 by Lytha Ayth during a long Emacs lovefest weekend in April, 2001.
The \acronym{UTF} is easier to search than the \acronym{HTML} format.
It is also much more accessible to people running on modest computers, such as donated '386-based PCs.
A 386 can, in theory, run Linux, Emacs, and a Scheme interpreter simultaneously, but most 386s probably can’t also run both Netscape and the necessary X Window System without prematurely introducing budding young underfunded hackers to the concept of \newterm{thrashing}.
\acronym{UTF} can also fit uncompressed on a 1.44\acronym{MB} floppy diskette, which may come in handy for installing \acronym{UTF} on PCs that do not have Internet or \acronym{LAN} access.
The Texinfo conversion has been a straight transliteration, to the extent possible.
Like the \TeX{}-to-\acronym{HTML} conversion, this was not without some introduction of breakage.
In the case of Unofficial Texinfo Format, figures have suffered an amateurish resurrection of the lost art of \acronym{ASCII}.
Also, it’s quite possible that some errors of ambiguity were introduced during the conversion of some of the copious superscripts (‘\^{}’) and subscripts (‘\_’).
Divining \emph{which} has been left as an exercise to the reader.
But at least we don’t put our brave astronauts at risk by encoding the \emph{greater-than-or-equal} symbol as \code{<u>\&gt;</u>}.
If you modify \texttt{sicp.texi} to correct errors or improve the \acronym{ASCII} art, then update the \code{@set utfversion 2.andresraba5.6} line to reflect your delta.
For example, if you started with Lytha’s version \code{1}, and your name is Bob, then you could name your successive versions \code{1.bob1}, \code{1.bob2}, …, \code{1.bob\( n \)}.
Also update \code{utfversiondate}.
If you want to distribute your version on the Web, then embedding the string “sicp.texi” somewhere in the file or Web page will make it easier for people to find with Web search engines.
It is believed that the Unofficial Texinfo Format is in keeping with the spirit of the graciously freely-distributed \acronym{HTML} version.
But you never know when someone’s armada of lawyers might need something to do, and get their shorts all in a knot over some benign little thing, so think twice before you use your full name or distribute Info, \acronym{DVI}, PostScript, or \acronym{PDF} formats that might embed your account or machine name.

\vspace{0.5em}
\noindent
\textit{Peath, Lytha Ayth}

\paragraph{Addendum:}
See also the \acronym{SICP} video lectures by Abelson and Sussman: at \href{http://groups.csail.mit.edu/mac/classes/6.001/abelson-sussman-lectures/}{\acronym{MIT CSAIL}} or
\href{http://ocw.mit.edu/courses/electrical-engineering-and-computer-science/6-001-structure-and-interpretation-of-computer-programs-spring-2005/video-lectures/}{\acronym{MIT OCW}}.

\paragraph{Second Addendum:}
Above is the original introduction to the \acronym{UTF} from 2001.
Ten years later, \acronym{UTF} has been transformed:
mathematical symbols and formulas are properly typeset, and figures drawn in vector graphics.
The original text formulas and \acronym{ASCII} art figures are still there in the Texinfo source, but will display only when compiled to Info output.
At the dawn of e-book readers and tablets, reading a \acronym{PDF} on screen is officially not silly anymore.
Enjoy!

\vspace{0.5em}
\noindent
\textit{A.R, May, 2011}
