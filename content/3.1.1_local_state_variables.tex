\subsection{Local State Variables}
\label{Section 3.1.1}

To illustrate what we mean by having a computational object with time-varying state, let us model the situation of withdrawing money from a bank account.
We will do this using a procedure \code{withdraw}, which takes as argument an \code{amount} to be withdrawn.
If there is enough money in the account to accommodate the withdrawal, then \code{withdraw} should return the balance remaining after the withdrawal.
Otherwise, \code{withdraw} should return the message \emph{Insufficient funds}.
For example, if we begin with \dollar{100} in the account, we should obtain the following sequence of responses using \code{withdraw}:
\begin{scheme}
  (withdraw 25)
  ~\outprint{75}~

  (withdraw 25)
  ~\outprint{50}~

  (withdraw 60)
  ~\outprint{"Insufficient funds"}~

  (withdraw 15)
  ~\outprint{35}~
\end{scheme}
Observe that the expression \code{(withdraw 25)}, evaluated twice, yields different values.
This is a new kind of behavior for a procedure.
Until now, all our procedures could be viewed as specifications for computing mathematical functions.
A call to a procedure computed the value of the function applied to the given arguments, and two calls to the same procedure with the same arguments always produced the same result.%
\footnote{
	Actually, this is not quite true.
	One exception was the random-number generator in \cref{Section 1.2.6}.
	Another exception involved the operation/type tables we introduced in \cref{Section 2.4.3}, where the values of two calls to \code{get} with the same arguments depended on intervening calls to \code{put}.
	On the other hand, until we introduce assignment, we have no way to create such procedures ourselves.
}

To implement \code{withdraw}, we can use a variable \code{balance} to indicate the balance of money in the account and define \code{withdraw} as a procedure that accesses \code{balance}.
The \code{withdraw} procedure checks to see if \code{balance} is at least as large as the requested \code{amount}.
If so, \code{withdraw} decrements \code{balance} by \code{amount} and returns the new value of \code{balance}.
Otherwise, \code{withdraw} returns the \emph{Insufficient funds} message.
Here are the definitions of \code{balance} and \code{withdraw}:
\begin{scheme}
  (define balance 100)

  (define (withdraw amount)
    (if (>= balance amount)
        (begin (set! balance (- balance amount))
               balance)
        "Insufficient funds"))
\end{scheme}
Decrementing \code{balance} is accomplished by the expression
\begin{scheme}
  (set! balance (- balance amount))
\end{scheme}
This uses the \code{set!} special form, whose syntax is
\begin{scheme}
  (set! ⟨~\var{name}~⟩ ⟨~\var{\dark new-value}~⟩)
\end{scheme}
Here \code{⟨\var{name}⟩} is a symbol and \code{⟨\var{new-value}⟩} is any expression.
\code{set!} changes \code{⟨\var{name}⟩} so that its value is the result obtained by evaluating \code{⟨\var{new-value}⟩}.
In the case at hand, we are changing \code{balance} so that its new value will be the result of subtracting \code{amount} from the previous value of \code{balance}.%
\footnote{
	The value of a \code{set!} expression is implementation-dependent.
	\code{set!} should be used only for its effect, not for its value.

	The name \code{set!} reflects a naming convention used in Scheme:
	Operations that change the values of variables (or that change data structures, as we will see in \cref{Section 3.3}) are given names that end with an exclamation point.
	This is similar to the convention of designating predicates by names that end with a question mark.
}

\code{withdraw} also uses the \code{begin} special form to cause two expressions to be evaluated in the case where the \code{if} test is true:
first decrementing \code{balance} and then returning the value of \code{balance}.
In general, evaluating the expression
\begin{scheme}
  (begin ⟨~\var{exp}\ind{1}~⟩ ⟨~\var{exp}\ind{2}~⟩ … ⟨~\var{exp}\ind{k}~⟩)
\end{scheme}
causes the expressions \code{⟨\var{exp}\ind{1}⟩} through \code{⟨\var{exp}\ind{k}⟩} to be evaluated in sequence and the value of the final expression \code{⟨\var{exp}\ind{k}⟩} to be returned as the value of the entire \code{begin} form.%
\footnote{
	We have already used \code{begin} implicitly in our programs, because in Scheme the body of a procedure can be a sequence of expressions.
	Also, the \code{⟨\var{consequent}⟩} part of each clause in a \code{cond} expression can be a sequence of expressions rather than a single expression.
}

Although \code{withdraw} works as desired, the variable \code{balance} presents a problem.
As specified above, \code{balance} is a name defined in the global environment and is freely accessible to be examined or modified by any procedure.
It would be much better if we could somehow make \code{balance} internal to \code{withdraw}, so that \code{withdraw} would be the only procedure that could access \code{balance} directly and any other procedure could access \code{balance} only indirectly (through calls to \code{withdraw}).
This would more accurately model the notion that \code{balance} is a local state variable used by \code{withdraw} to keep track of the state of the account.

We can make \code{balance} internal to \code{withdraw} by rewriting the definition as follows:
\begin{scheme}
  (define new-withdraw
    (let ((balance 100))
      (lambda (amount)
        (if (>= balance amount)
            (begin (set! balance (- balance amount))
                   balance)
            "Insufficient funds"))))
\end{scheme}
What we have done here is use \code{let} to establish an environment with a local variable \code{balance}, bound to the initial value \( 100 \).
Within this local environment, we use \code{lambda} to create a procedure that takes \code{amount} as an argument and behaves like our previous \code{withdraw} procedure.
This procedure---returned as the result of evaluating the \code{let} expression---is \code{new-withdraw}, which behaves in precisely the same way as \code{withdraw} but whose variable \code{balance} is not accessible by any other procedure.%
\footnote{
	In programming-language jargon, the variable \code{balance} is said to be \newterm{encapsulated} within the \code{new-withdraw} procedure.
	Encapsulation reflects the general system-design principle known as the \newterm{hiding principle}:
	One can make a system more modular and robust by protecting parts of the system from each other;
	that is, by providing information access only to those parts of the system that have a “need to know.”
}

Combining \code{set!} with local variables is the general programming technique we will use for constructing computational objects with local state.
Unfortunately, using this technique raises a serious problem:
When we first introduced procedures, we also introduced the substitution model of evaluation (\cref{Section 1.1.5}) to provide an interpretation of what procedure application means.
We said that applying a procedure should be interpreted as evaluating the body of the procedure with the formal parameters replaced by their values.
The trouble is that, as soon as we introduce assignment into our language, substitution is no longer an adequate model of procedure application.
(We will see why this is so in \cref{Section 3.1.3}.)
As a consequence, we technically have at this point no way to understand why the \code{new-withdraw} procedure behaves as claimed above.
In order to really understand a procedure such as \code{new-withdraw}, we will need to develop a new model of procedure application.
In \cref{Section 3.2} we will introduce such a model, together with an explanation of \code{set!} and local variables.
First, however, we examine some variations on the theme established by \code{new-withdraw}.

The following procedure, \code{make-withdraw}, creates “withdrawal processors.”
The formal parameter \code{balance} in \code{make-withdraw} specifies the initial amount of money in the account.%
\footnote{
	In contrast with \code{new-withdraw} above, we do not have to use \code{let} to make \code{balance} a local variable, since formal parameters are already local.
	This will be clearer after the discussion of the environment model of evaluation in \cref{Section 3.2}.
	(See also \cref{Exercise 3.10}.)
}
\begin{scheme}
  (define (make-withdraw balance)
    (lambda (amount)
      (if (>= balance amount)
          (begin (set! balance (- balance amount))
                 balance)
          "Insufficient funds")))
\end{scheme}

\code{make-withdraw} can be used as follows to create two objects \code{W1} and \code{W2}:
\begin{scheme}
  (define W1 (make-withdraw 100))
  (define W2 (make-withdraw 100))

  (W1 50)
  ~\outprint{50}~

  (W2 70)
  ~\outprint{30}~

  (W2 40)
  ~\outprint{"Insufficient funds"}~

  (W1 40)
  ~\outprint{10}~
\end{scheme}

\noindent
Observe that \code{W1} and \code{W2} are completely independent objects, each with its own local state variable \code{balance}.
Withdrawals from one do not affect the other.

We can also create objects that handle deposits as well as withdrawals, and thus we can represent simple bank accounts.
Here is a procedure that returns a “bank-account object” with a specified initial balance:
\begin{scheme}
  (define (make-account balance)
    (define (withdraw amount)
      (if (>= balance amount)
          (begin (set! balance (- balance amount))
                 balance)
          "Insufficient funds"))
    (define (deposit amount)
      (set! balance (+ balance amount))
      balance)
    (define (dispatch m)
      (cond ((eq? m 'withdraw) withdraw)
            ((eq? m 'deposit) deposit)
            (else (error "Unknown request: MAKE-ACCOUNT"
                         m))))
    dispatch)
\end{scheme}
Each call to \code{make-account} sets up an environment with a local state variable \code{balance}.
Within this environment, \code{make-account} defines procedures \code{deposit} and \code{withdraw} that access \code{balance} and an additional procedure \code{dispatch} that takes a “message” as input and returns one of the two local procedures.
The \code{dispatch} procedure itself is returned as the value that represents the bank-account object.
This is precisely the \newterm{message-passing} style of programming that we saw in \cref{Section 2.4.3}, although here we are using it in conjunction with the ability to modify local variables.

\code{make-account} can be used as follows:
\begin{scheme}
  (define acc (make-account 100))

  ((acc 'withdraw) 50)
  ~\outprint{50}~

  ((acc 'withdraw) 60)
  ~\outprint{"Insufficient funds"}~

  ((acc 'deposit) 40)
  ~\outprint{90}~

  ((acc 'withdraw) 60)
  ~\outprint{30}~
\end{scheme}
Each call to \code{acc} returns the locally defined \code{deposit} or \code{withdraw} procedure, which is then applied to the specified \code{amount}.
As was the case with \code{make-withdraw}, another call to \code{make-account}
\begin{scheme}
  (define acc2 (make-account 100))
\end{scheme}
will produce a completely separate account object, which maintains its own local \code{balance}.



\begin{exercise}
	\label{Exercise 3.1}
	An \newterm{accumulator} is a procedure that is called repeatedly with a single numeric argument and accumulates its arguments into a sum.
	Each time it is called, it returns the currently accumulated sum.
	Write a procedure \code{make-accumulator} that generates accumulators, each maintaining an independent sum.
	The input to \code{make-accumulator} should specify the initial value of the sum;
	for example
	\begin{scheme}
	  (define A (make-accumulator 5))

	  (A 10)
	  ~\outprint{15}~

	  (A 10)
	  ~\outprint{25}~
	\end{scheme}
\end{exercise}



\begin{exercise}
	\label{Exercise 3.2}
	In software-testing applications, it is useful to be able to count the number of times a given procedure is called during the course of a computation.
	Write a procedure \code{make-monitored} that takes as input a procedure, \code{f}, that itself takes one input.
	The result returned by \code{make-monitored} is a third procedure, say \code{mf}, that keeps track of the number of times it has been called by maintaining an internal counter.
	If the input to \code{mf} is the special symbol \code{how-many-calls?}, then \code{mf} returns the value of the counter.
	If the input is the special symbol \code{reset-count}, then \code{mf} resets the counter to zero.
	For any other input, \code{mf} returns the result of calling \code{f} on that input and increments the counter.
	For instance, we could make a monitored version of the \code{sqrt} procedure:
	\begin{scheme}
	  (define s (make-monitored sqrt))

	  (s 100)
	  ~\outprint{10}~

	  (s 'how-many-calls?)
	  ~\outprint{1}~
	\end{scheme}
\end{exercise}



\begin{exercise}
	\label{Exercise 3.3}
	Modify the \code{make-account} procedure so that it creates password-protected accounts.
	That is, \code{make-account} should take a symbol as an additional argument, as in
	\begin{scheme}
	  (define acc (make-account 100 'secret-password))
	\end{scheme}
	The resulting account object should process a request only if it is accompanied by the password with which the account was created, and should otherwise return a complaint:
	\begin{scheme}
	  ((acc 'secret-password 'withdraw) 40)
	  ~\outprint{60}~

	  ((acc 'some-other-password 'deposit) 50)
	  ~\outprint{"Incorrect password"}~
	\end{scheme}
\end{exercise}



\begin{exercise}
	\label{Exercise 3.4}
	Modify the \code{make-account} procedure of \cref{Exercise 3.3} by adding another local state variable so that, if an account is accessed more than seven consecutive times with an incorrect password, it invokes the procedure \code{call-the-cops}.
\end{exercise}
