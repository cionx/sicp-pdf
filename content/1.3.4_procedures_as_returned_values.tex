\subsection{Procedures as Returned Values}
\label{Section 1.3.4}

The above examples demonstrate how the ability to pass procedures as arguments significantly enhances the expressive power of our programming language.
We can achieve even more expressive power by creating procedures whose returned values are themselves procedures.

We can illustrate this idea by looking again at the fixed-point example described at the end of \cref{Section 1.3.3}.
We formulated a new version of the square-root procedure as a fixed-point search, starting with the observation that \( \sqrt{x} \) is a fixed-point of the function \( y \mapsto x / y \).
Then we used average damping to make the approximations converge.
Average damping is a useful general technique in itself.
Namely, given a function \( f \), we consider the function whose value at \( x \) is equal to the average of \( x \) and \( f(x) \).

We can express the idea of average damping by means of the following procedure:
\begin{scheme}
  (define (average-damp f)
    (lambda (x) (average x (f x))))
\end{scheme}
\code{average-damp} is a procedure that takes as its argument a procedure \code{f} and returns as its value a procedure (produced by the \code{lambda}) that, when applied to a number \code{x}, produces the average of \code{x} and \code{(f x)}.
For example, applying \code{average-damp} to the \code{square} procedure produces a procedure whose value at some number \( x \) is the average of \( x \) and \( x^2 \).
Applying this resulting procedure to \( 10 \) returns the average of \( 10 \) and \( 100 \), or \( 55 \):%
\footnote{
	Observe that this is a combination whose operator is itself a combination.
	\cref{Exercise 1.4} already demonstrated the ability to form such combinations, but that was only a toy example.
	Here we begin to see the real need for such combinations---when applying a procedure that is obtained as the value returned by a higher-order procedure.
}
\begin{scheme}
  ((average-damp square) 10)
  ~\outprint{55}~
\end{scheme}

Using \code{average-damp}, we can reformulate the square-root procedure as follows:
\begin{scheme}
  (define (sqrt x)
    (fixed-point (average-damp (lambda (y) (/ x y)))
                 1.0))
\end{scheme}
Notice how this formulation makes explicit the three ideas in the method:
fixed-point search, average damping, and the function \( y \mapsto x / y \).
It is instructive to compare this formulation of the square-root method with the original version given in \cref{Section 1.1.7}.
Bear in mind that these procedures express the same process, and notice how much clearer the idea becomes when we express the process in terms of these abstractions.
In general, there are many ways to formulate a process as a procedure.
Experienced programmers know how to choose procedural formulations that are particularly perspicuous, and where useful elements of the process are exposed as separate entities that can be reused in other applications.
As a simple example of reuse, notice that the cube root of \( x \) is a fixed point of the function \( y \mapsto x / y^2 \), so we can immediately generalize our square-root procedure to one that extracts cube roots:%
\footnote{
	See \cref{Exercise 1.45} for a further generalization.
}
\begin{scheme}
  (define (cube-root x)
    (fixed-point (average-damp (lambda (y) (/ x (square y))))
                 1.0))
\end{scheme}



\subsubsection*{Newton’s method}

When we first introduced the square-root procedure, in \cref{Section 1.1.7}, we mentioned that this was a special case of \newterm{Newton’s method}.
If \( x \mapsto g(x) \) is a differentiable function, then a solution of the equation \( g(x) = 0 \) is a fixed point of the function \( x \mapsto f(x) \), where
\[
	f(x) = x - \frac{g(x)}{Dg(x)}
\]
and \( Dg(x) \) is the derivative of \( g \) evaluated at \( x \).
Newton’s method is the use of the fixed-point method we saw above to approximate a solution of the equation by finding a fixed point of the function \( f \).%
\footnote{
	Elementary calculus books usually describe Newton’s method in terms of the sequence of approximations \( x_{n+1} = x_n - g(x_n) / Dg(x_n) \).
	Having language for talking about processes and using the idea of fixed points simplifies the description of the method.
}
For many functions \( g \) and for sufficiently good initial guesses for \( x \),
Newton’s method converges very rapidly to a solution of \( g(x) = 0 \).%
\footnote{
	Newton’s method does not always converge to an answer, but it can be shown that in favorable cases each iteration doubles the number-of-digits accuracy of the approximation to the solution.
	In such cases, Newton’s method will converge much more rapidly than the half-interval method.
}

In order to implement Newton’s method as a procedure, we must first express the idea of derivative.
Note that “derivative,” like average damping, is something that transforms a function into another function.
For instance, the derivative of the function \( x \mapsto x^3 \) is the function \( x \mapsto 3x^2 \).
In general, if \( g \) is a function and \( dx \) is a small number, then the derivative \( Dg \) of \( g \) is the function whose value at any number \( x \) is given (in the limit of small \( dx \)) by
\[
	Dg(x) ≈ \frac{g(x + dx) - g(x)}{dx}\,.
\]
Thus, we can express the idea of derivative (taking \( dx \) to be, say,
\( 0.00001 \)) as the procedure
\begin{scheme}
  (define (deriv g)
    (lambda (x) (/ (- (g (+ x dx)) (g x)) dx)))
\end{scheme}
along with the definition
\begin{scheme}
  (define dx 0.00001)
\end{scheme}

Like \code{average-damp}, \code{deriv} is a procedure that takes a procedure as argument and returns a procedure as value.
For example, to approximate the derivative of \( x \mapsto x^3 \) at \( 5 \) (whose exact value is \( 75 \)) we can evaluate
\begin{scheme}
  (define (cube x) (* x x x))
  ((deriv cube) 5)
  ~\outprint{75.00014999664018}~
\end{scheme}

With the aid of \code{deriv}, we can express Newton’s method as a fixed-point process:
\begin{scheme}
  (define (newton-transform g)
    (lambda (x) (- x (/ (g x) ((deriv g) x)))))

  (define (newtons-method g guess)
    (fixed-point (newton-transform g) guess))
\end{scheme}
The \code{newton-transform} procedure expresses the formula at the beginning of this section, and \code{newtons-method} is readily defined in terms of this.
It takes as arguments a procedure that computes the function for which we want to find a zero, together with an initial guess.
For instance, to find the square root of \( x \), we can use Newton’s method to find a zero of the function \( y \mapsto y^2 - x \) starting with an initial guess of \( 1 \).%
\footnote{
	For finding square roots, Newton’s method converges rapidly to the correct solution from any starting point.
}
This provides yet another form of the square-root procedure:
\begin{scheme}
  (define (sqrt x)
    (newtons-method (lambda (y) (- (square y) x))
                    1.0))
\end{scheme}



\subsubsection*{Abstractions and first-class procedures}

We’ve seen two ways to express the square-root computation as an instance of a more general method, once as a fixed-point search and once using Newton’s method.
Since Newton’s method was itself expressed as a fixed-point process, we actually saw two ways to compute square roots as fixed points.
Each method begins with a function and finds a fixed point of some transformation of the function.
We can express this general idea itself as a procedure:
\begin{scheme}
  (define (fixed-point-of-transform g transform guess)
    (fixed-point (transform g) guess))
\end{scheme}
This very general procedure takes as its arguments a procedure \code{g} that computes some function, a procedure that transforms \code{g}, and an initial guess.
The returned result is a fixed point of the transformed function.

Using this abstraction, we can recast the first square-root computation from this section (where we look for a fixed point of the average-damped version of \( y \mapsto x / y \)) as an instance of this general method:
\begin{scheme}
  (define (sqrt x)
    (fixed-point-of-transform
     (lambda (y) (/ x y)) average-damp 1.0))
\end{scheme}
Similarly, we can express the second square-root computation from this section (an instance of Newton’s method that finds a fixed point of the Newton transform of \( y \mapsto y^2 - x \)) as
\begin{scheme}
  (define (sqrt x)
    (fixed-point-of-transform
     (lambda (y) (- (square y) x)) newton-transform 1.0))
\end{scheme}

We began \cref{Section 1.3} with the observation that compound procedures are a crucial abstraction mechanism, because they permit us to express general methods of computing as explicit elements in our programming language.
Now we’ve seen how higher-order procedures permit us to manipulate these general methods to create further abstractions.

As programmers, we should be alert to opportunities to identify the underlying abstractions in our programs and to build upon them and generalize them to create more powerful abstractions.
This is not to say that one should always write programs in the most abstract way possible;
expert programmers know how to choose the level of abstraction appropriate to their task.
But it is important to be able to think in terms of these abstractions, so that we can be ready to apply them in new contexts.
The significance of higher-order procedures is that they enable us to represent these abstractions explicitly as elements in our programming language, so that they can be handled just like other computational elements.

In general, programming languages impose restrictions on the ways in which computational elements can be manipulated.
Elements with the fewest restrictions are said to have \newterm{first-class} status.
Some of the “rights and privileges” of first-class elements are:%
\footnote{
	The notion of first-class status of programming-language elements is due to the British computer scientist Christopher Strachey (1916--1975).
}
\begin{itemize}

	\item
		They may be named by variables.

	\item
		They may be passed as arguments to procedures.

	\item
		They may be returned as the results of procedures.

	\item
		They may be included in data structures.%
		\footnote{
			We’ll see examples of this after we introduce data structures in \cref{Chapter 2}.
		}

\end{itemize}
Lisp, unlike other common programming languages, awards procedures full first-class status.
This poses challenges for efficient implementation, but the resulting gain in expressive power is enormous.%
\footnote{
	The major implementation cost of first-class procedures is that allowing procedures to be returned as values requires reserving storage for a procedure’s free variables even while the procedure is not executing.
	In the Scheme implementation we will study in \cref{Section 4.1}, these variables are stored in the procedure’s environment.
}



\begin{exercise}
	\label{Exercise 1.40}
	Define a procedure \code{cubic} that can be used together with the \code{newtons-method} procedure in expressions of the form
	\begin{scheme}
	  (newtons-method (cubic a b c) 1)
	\end{scheme}
	to approximate zeros of the cubic \( x^3 + ax^2 + bx + c \).
\end{exercise}



\begin{exercise}
	\label{Exercise 1.41}
	Define a procedure \code{double} that takes a procedure of one argument as argument and returns a procedure that applies the original procedure twice.
	For example, if \code{inc} is a procedure that adds \( 1 \) to its argument, then \code{(double inc)} should be a procedure that adds \( 2 \).
	What value is returned by
	\begin{scheme}
	  (((double (double double)) inc) 5)
	\end{scheme}
\end{exercise}



\begin{exercise}
	\label{Exercise 1.42}
	Let \( f \) and \( g \) be two one-argument functions.
	The \newterm{composition} \( f \) after \( g \) is defined to be the function \( x \mapsto f(g(x)) \).
	Define a procedure \code{compose} that implements composition.
	For example, if \code{inc} is a procedure that adds \( 1 \) to its argument,
	\begin{scheme}
	  ((compose square inc) 6)
	  ~\outprint{49}~
	\end{scheme}
\end{exercise}



\begin{exercise}
	\label{Exercise 1.43}
	If \( f \) is a numerical function and \( n \) is a positive integer, then we can form the \( n \)\nth{th} repeated application of \( f \), which is defined to be the function whose value at \( x \) is \( f(f( \dotsb f(x) \dotsb )) \).
	For example, if \( f \) is the function \( x \mapsto x + 1 \), then the \( n \)\nth{th} repeated application of \( f \) is the function \( x \mapsto x + n \).
	If \( f \) is the operation of squaring a number, then the \( n \)\nth{th} repeated application of \( f \) is the function that raises its argument to the (\( 2^n \))\nth{th} power.
	Write a procedure \code{repeated} that takes as inputs a procedure that computes \( f \) and a positive integer \( n \) and returns the procedure that computes the \( n \)\nth{th} repeated application of \( f \).
	Your procedure should be able to be used as follows:
	\begin{scheme}
	  ((repeated square 2) 5)
	  ~\outprint{625}~
	\end{scheme}
	Hint:
	You may find it convenient to use \code{compose} from \cref{Exercise 1.42}.
\end{exercise}



\begin{exercise}
	\label{Exercise 1.44}
	The idea of \newterm{smoothing} a function is an important concept in signal processing.
	If \( f \) is a function and \( dx \) is some small number, then the smoothed version of \( f \) is the function whose value at a point \( x \) is the average of \( f(x - dx) \), \( f(x) \), and \( f(x + dx) \).
	Write a procedure \code{smooth} that takes as input a procedure that computes \( f \) and returns a procedure that computes the smoothed \( f \).
	It is sometimes valuable to repeatedly smooth a function (that is, smooth the smoothed function, and so on) to obtain the \newterm{\( n \)-fold smoothed function}.
	Show how to generate the \( n \)-fold smoothed function of any given function using \code{smooth} and \code{repeated} from \cref{Exercise 1.43}.
\end{exercise}



\begin{exercise}
	\label{Exercise 1.45}
	We saw in \cref{Section 1.3.3} that attempting to compute square roots by naively finding a fixed point of \( y \mapsto x / y \) does not converge, and that this can be fixed by average damping.
	The same method works for finding cube roots as fixed points of the average-damped \( y \mapsto x / y^2 \).
	Unfortunately, the process does not work for fourth roots---a single average damp is not enough to make a fixed-point search for \( y \mapsto x / y^3 \) converge.
	On the other hand, if we average damp twice (i.e., use the average damp of the average damp of \( y \mapsto x / y^3 \)) the fixed-point search does converge.
	Do some experiments to determine how many average damps are required to compute \( n \)\nth{th} roots as a fixed-point search based upon repeated average damping of \( y \mapsto x / y^{n - 1} \).
	Use this to implement a simple procedure for computing \( n \)\nth{th} roots using \code{fixed-point}, \code{average-damp}, and the \code{repeated} procedure of \cref{Exercise 1.43}.
	Assume that any arithmetic operations you need are available as primitives.
\end{exercise}



\begin{exercise}
	\label{Exercise 1.46}
	Several of the numerical methods described in this chapter are instances of an extremely general computational strategy known as \newterm{iterative improvement}.
	Iterative improvement says that, to compute something, we start with an initial guess for the answer, test if the guess is good enough, and otherwise improve the guess and continue the process using the improved guess as the new guess.
	Write a procedure \code{iterative-improve} that takes two procedures as arguments:
	a method for telling whether a guess is good enough and a method for improving a guess.
	\code{iterative-improve} should return as its value a procedure that takes a guess as argument and keeps improving the guess until it is good enough.
	Rewrite the \code{sqrt} procedure of \cref{Section 1.1.7} and the \code{fixed-point} procedure of \cref{Section 1.3.3} in terms of \code{iterative-improve}.
\end{exercise}
