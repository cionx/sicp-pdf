\subsection{Instruction Summary}
\label{Section 5.1.5}

A controller instruction in our register-machine language has one of the following forms, where each \code{⟨\var{input}\ind{i}⟩} is either \code{(reg ⟨\var{register-name}⟩)} or \code{(const ⟨\var{constant-value}⟩)}.
These instructions were introduced in \cref{Section 5.1.1}:
\begin{scheme}
  (assign ⟨~\var{\dark register-name}~⟩ (reg ⟨~\var{\dark register-name}~⟩))

  (assign ⟨~\var{\dark register-name}~⟩ (const ⟨~\var{\dark constant-value}~⟩))

  (assign ⟨~\var{\dark register-name}~⟩
          (op ⟨~\var{\dark operation-name}~⟩)
          ⟨~\var{input}\ind{1}~⟩ … ⟨~\var{input}\ind{n}~⟩)

  (perform (op ⟨~\var{\dark operation-name}~⟩) ⟨~\var{input}\ind{1}~⟩ … ⟨~\var{input}\ind{n}~⟩)

  (test (op ⟨~\var{\dark operation-name}~⟩) ⟨~\var{input}\ind{1}~⟩ … ⟨~\var{input}\ind{n}~⟩)

  (branch (label ⟨~\var{\dark label-name}~⟩))

  (goto (label ⟨~\var{\dark label-name}~⟩))
\end{scheme}

The use of registers to hold labels was introduced in \cref{Section 5.1.3}:
\begin{scheme}
  (assign ⟨~\var{\dark register-name}~⟩ (label ⟨~\var{\dark label-name}~⟩))

  (goto (reg ⟨~\var{\dark register-name}~⟩))
\end{scheme}

Instructions to use the stack were introduced in \cref{Section 5.1.4}:
\begin{scheme}
  (save ⟨~\var{\dark register-name}~⟩)

  (restore ⟨~\var{\dark register-name}~⟩)
\end{scheme}

The only kind of ⟨\var{constant-value}⟩ we have seen so far is a number, but later we will use strings, symbols, and lists.
For example,
\begin{itemize}

	\item
		\code{(const "abc")} is the string \code{"abc"},

	\item
		\code{(const abc)} is the symbol \code{abc},

	\item
		\code{(const (a b c))} is the list \code{(a b c)},

	\item
		\code{(const ())} is the empty list.

\end{itemize}
