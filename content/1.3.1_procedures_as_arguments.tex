\subsection{Procedures as Arguments}
\label{Section 1.3.1}

Consider the following three procedures.
The first computes the sum of the integers from \code{a} through \code{b}:
\begin{scheme}
  (define (sum-integers a b)
    (if (> a b)
        0
        (+ a (sum-integers (+ a 1) b))))
\end{scheme}
The second computes the sum of the cubes of the integers in the given range:
\begin{scheme}
  (define (sum-cubes a b)
    (if (> a b)
        0
        (+ (cube a)
           (sum-cubes (+ a 1) b))))
\end{scheme}
The third computes the sum of a sequence of terms in the series
\[
	\frac{1}{1 ⋅ 3} + \frac{1}{5 ⋅ 7} + \frac{1}{9 ⋅ 11} + \dotsb
\]
which converges to \( π / 8 \) (very slowly):%
\footnote{
	This series, usually written in the equivalent form $π/4 = 1 - 1/3 + 1/5 - 1/7 + \dotsb \,,$ is due to Leibniz.
	We’ll see how to use this as the basis for some fancy numerical tricks in \cref{Section 3.5.3}.
}
\begin{scheme}
  (define (pi-sum a b)
    (if (> a b)
        0
        (+ (/ 1.0 (* a (+ a 2)))
           (pi-sum (+ a 4) b))))
\end{scheme}

These three procedures clearly share a common underlying pattern.
They are for the most part identical, differing only in the name of the procedure, the function of \code{a} used to compute the term to be added, and the function that provides the next value of \code{a}.
We could generate each of the procedures by filling in slots in the same template:
\begin{scheme}
  (define (~\cproc{⟨name⟩}~ a b)
    (if (> a b)
        0
        (+ (~\cvar{⟨term⟩}~ a)
           (~\cvar{⟨name⟩}~ (~\cvar{⟨next⟩}~ a) b))))
\end{scheme}

The presence of such a common pattern is strong evidence that there is a useful abstraction waiting to be brought to the surface.
Indeed, mathematicians long ago identified the abstraction of \newterm{summation of a series} and invented “sigma notation,” for example
\[
	∑_{n = a}^b f(n) = f(a) + \dotsb + f(b) \,,
\]
to express this concept.
The power of sigma notation is that it allows mathematicians to deal with the concept of summation itself rather than only with particular sums---for example, to formulate general results about sums that are independent of the particular series being summed.

Similarly, as program designers, we would like our language to be powerful enough so that we can write a procedure that expresses the concept of summation itself rather than only procedures that compute particular sums.
We can do so readily in our procedural language by taking the common template shown above and transforming the “slots” into formal parameters:
\begin{scheme}
  (define (sum term a next b)
    (if (> a b)
        0
        (+ (term a)
           (sum term (next a) next b))))
\end{scheme}
Notice that \code{sum} takes as its arguments the lower and upper bounds \code{a} and \code{b} together with the procedures \code{term} and \code{next}.
We can use \code{sum} just as we would any procedure.
For example, we can use it (along with a procedure \code{inc} that increments its argument by \( 1 \)) to define \code{sum-cubes}:
\begin{scheme}
  (define (inc n) (+ n 1))

  (define (sum-cubes a b)
    (sum cube a inc b))
\end{scheme}
Using this, we can compute the sum of the cubes of the integers from \( 1 \) to \( 10 \):
\begin{scheme}
  (sum-cubes 1 10)
  ~\outprint{3025}~
\end{scheme}
With the aid of an identity procedure to compute the term, we can define
\code{sum-integers} in terms of \code{sum}:
\begin{scheme}
  (define (identity x) x)

  (define (sum-integers a b)
    (sum identity a inc b))
\end{scheme}
Then we can add up the integers from \( 1 \) to \( 10 \):
\begin{scheme}
  (sum-integers 1 10)
  ~\outprint{55}~
\end{scheme}
We can also define \code{pi-sum} in the same way:%
\footnote{
	Notice that we have used block structure (\cref{Section 1.1.8}) to embed the definitions of \code{pi-next} and \code{pi-term} within \code{pi-sum}, since these procedures are unlikely to be useful for any other purpose.
	We will see how to get rid of them altogether in \cref{Section 1.3.2}.
}
\begin{scheme}
  (define (pi-sum a b)
    (define (pi-term x)
      (/ 1.0 (* x (+ x 2))))
    (define (pi-next x)
      (+ x 4))
    (sum pi-term a pi-next b))
\end{scheme}
Using these procedures, we can compute an approximation to \( π \):
\begin{scheme}
  (* 8 (pi-sum 1 1000))
  ~\outprint{3.139592655589783}~
\end{scheme}

Once we have \code{sum}, we can use it as a building block in formulating further concepts.
For instance, the definite integral of a function \( f \) between the limits \( a \) and \( b \) can be approximated numerically using the formula
\[
	\int_a^b f
	=
	\biggl[
		f\biggl( a + \frac{dx}{2} \biggr)
		+ f\biggl( a + dx + \frac{dx}{2} \biggr)
		+ f\biggl( a + 2 dx + \frac{dx}{2} \biggr)
		+ \dotsb
	\biggr]
	dx
\]
for small values of \( dx \).
We can express this directly as a procedure:
\begin{scheme}
  (define (integral f a b dx)
    (define (add-dx x)
      (+ x dx))
    (* (sum f (+ a (/ dx 2.0)) add-dx b)
       dx))
\end{scheme}
\begin{scheme}
  (integral cube 0 1 0.01)
  ~\outprint{.24998750000000042}~

  (integral cube 0 1 0.001)
  ~\outprint{.249999875000001}~
\end{scheme}
(The exact value of the integral of \code{cube} between \( 0 \) and \( 1 \) is \( 1/4 \).)



\begin{exercise}
	\label{Exercise 1.29}
	Simpson’s rule is a more accurate method of numerical integration than the method illustrated above.
	Using Simpson’s rule, the integral of a function \( f \) between \( a \) and \( b \) is approximated as
	\[
		\frac{h}{3}
		(y_0 + 4 y_1 + 2 y_2 + 4 y_3 + 2 y_4 + \dotsb + 2 y_{n-2} + 4 y_{n-1} + y_n) \,,
	\]
	where \( h = (b - a) / n \), for some even integer \( n \), and \( y_k = f(a + kh) \).
	(Increasing \( n \) increases the accuracy of the approximation.)
	Define a procedure that takes as arguments \( f \), \( a \), \( b \), and \( n \) and returns the value of the integral, computed using Simpson’s rule.
	Use your procedure to integrate \code{cube} between \( 0 \) and \( 1 \) (with \( n = 100 \) and \( n = 1000 \)), and compare the results to those of the \code{integral} procedure shown above.
\end{exercise}



\begin{exercise}
	\label{Exercise 1.30}
	The \code{sum} procedure above generates a linear recursion.
	The procedure can be rewritten so that the sum is performed iteratively.
	Show how to do this by filling in the missing expressions in the following definition:
	\begin{scheme}
	  (define (sum term a next b)
	    (define (iter a result)
	      (if ~\cvar{⟨??⟩}~
	          ~\cvar{⟨??⟩}~
	          (iter ~\cvar{⟨??⟩}~ ~\cvar{⟨??⟩}~)))
	    (iter ~\cvar{⟨??⟩}~ ~\cvar{⟨??⟩}~))
	\end{scheme}
\end{exercise}



\begin{exercise}
	\label{Exercise 1.31}
	\begin{enumerate}[label = \alph*., leftmargin = *]

		\item
			The \code{sum} procedure is only the simplest of a vast number of similar abstractions that can be captured as higher-order procedures.%
			\footnote{
				The intent of \cref{Exercise 1.31} through \cref{Exercise 1.33} is to demonstrate the expressive power that is attained by using an appropriate abstraction to consolidate many seemingly disparate operations.
				However, though accumulation and filtering are elegant ideas, our hands are somewhat tied in using them at this point since we do not yet have data structures to provide suitable means of combination for these abstractions.
				We will return to these ideas in \cref{Section 2.2.3} when we show how to use \newterm{sequences} as interfaces for combining filters and accumulators to build even more powerful abstractions.
				We will see there how these methods really come into their own as a powerful and elegant approach to designing programs.
			}
			Write an analogous procedure called \code{product} that returns the product of the values of a function at points over a given range.
			Show how to define \code{factorial} in terms of \code{product}.
			Also use \code{product} to compute approximations to \( π \) using the formula%
			\footnote{
				This formula was discovered by the seventeenth-century English mathematician John Wallis.
			}
			\[
				\frac{π}{4}
				=
				\frac{2 ⋅ 4 ⋅ 4 ⋅ 6 ⋅ 6 ⋅ 8 \dotsm}{3 ⋅ 3 ⋅ 5 ⋅ 5 ⋅ 7 ⋅ 7 \dotsm} \,.
			\]

		\item
			If your \code{product} procedure generates a recursive process, write one that generates an iterative process.
			If it generates an iterative process, write one that generates a recursive process.

	\end{enumerate}
\end{exercise}



\begin{exercise}
	\label{Exercise 1.32}
	\begin{enumerate}[label = \alph*., leftmargin = *]

		\item
			Show that \code{sum} and \code{product} (\cref{Exercise 1.31}) are both special cases of a still more general notion called \code{accumulate} that combines a collection of terms, using some general accumulation function:
			\begin{scheme}
			  (accumulate combiner null-value term a next b)
			\end{scheme}

			\code{accumulate} takes as arguments the same term and range specifications as \code{sum} and \code{product}, together with a \code{combiner} procedure (of two arguments) that specifies how the current term is to be combined with the accumulation of the preceding terms and a \code{null-value} that specifies what base value to use when the terms run out.
			Write \code{accumulate} and show how \code{sum} and \code{product} can both be defined as simple calls to \code{accumulate}.

		\item
			If your \code{accumulate} procedure generates a recursive process, write one that generates an iterative process.
			If it generates an iterative process, write one that generates a recursive process.

	\end{enumerate}
\end{exercise}



\begin{exercise}
	\label{Exercise 1.33}
	You can obtain an even more general version of \code{accumulate} (\cref{Exercise 1.32}) by introducing the notion of a \newterm{filter} on the terms to be combined.
	That is, combine only those terms derived from values in the range that satisfy a specified condition.
	The resulting \code{filtered-accumulate} abstraction takes the same arguments as accumulate, together with an additional predicate of one argument that specifies the filter.
	Write \code{filtered-accumulate} as a procedure.
	Show how to express the following using \code{filtered-accumulate}:
	\begin{enumerate}[label = \alph*., leftmargin = *]

		\item
			the sum of the squares of the prime numbers in the interval \( a \) to \( b \) (assuming that you have a \code{prime?} predicate already written).

		\item
			the product of all the positive integers less than \( n \) that are relatively prime to \( n \) (i.e., all positive integers \( i < n \) such that \( \gcd(i, n) = 1 \)).

	\end{enumerate}
\end{exercise}
