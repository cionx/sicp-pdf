\subsection{Quotation}
\label{Section 2.3.1}

If we can form compound data using symbols, we can have lists such as
\begin{scheme}
  (a b c d)
  (23 45 17)
  ((Norah 12) (Molly 9) (Anna 7) (Lauren 6) (Charlotte 4))
\end{scheme}
Lists containing symbols can look just like the expressions of our  language:

\begin{scheme}
  (* (+ 23 45)
     (+ x 9))

  (define (fact n)
    (if (= n 1) 1 (* n (fact (- n 1)))))
\end{scheme}

In order to manipulate symbols we need a new element in our language:
the ability to \newterm{quote} a data object.
Suppose we want to construct the list \code{(a b)}.
We can’t accomplish this with \code{(list a b)}, because this expression constructs a list of the \newterm{values} of \code{a} and \code{b} rather than the symbols themselves.
This issue is well known in the context of natural languages, where words and sentences may be regarded either as semantic entities or as character strings (syntactic entities).
The common practice in natural languages is to use quotation marks to indicate that a word or a sentence is to be treated literally as a string of characters.
For instance, the first letter of “John”  is clearly  “J.”
If we tell somebody “say your name aloud,” we expect to hear that person’s name.
However, if we tell somebody “say ‘your name’ aloud,” we expect to hear the words “your name.”
Note that we are forced to nest quotation marks to describe what somebody else might say.%
\footnote{
	Allowing quotation in a language wreaks havoc with the ability to reason about the language in simple terms, because it destroys the notion that equals can be substituted for equals.
	For example, three is one plus two, but the word “three” is not the phrase “one plus two.”
	Quotation is powerful because it gives us a way to build expressions that manipulate other expressions (as we will see when we write an interpreter in \link{Chapter 4}).
	But allowing statements in a language that talk about other statements in that language makes it very difficult to maintain any coherent principle of what “equals can be substituted for equals” should mean.
	For example, if we know that the evening star is the morning star, then from the statement “the evening star is Venus” we can deduce “the morning star is Venus.”
	However, given that  “John knows that the evening star is Venus” we cannot infer that “John knows that the morning star is Venus.”
}

We can follow this same practice to identify lists and symbols that are to be treated as data objects rather than as expressions to be evaluated.
However, our format for quoting differs from that of natural languages in that we place a quotation mark (traditionally, the single quote symbol \code{'}) only at the beginning of the object to be quoted.
We can get away with this in Scheme syntax because we rely on blanks and parentheses to delimit objects.
Thus, the meaning of the single quote character is to quote the next object.%
\footnote{
	The single quote is different from the double quote we have been using to enclose character strings to be printed.
	Whereas the single quote can be used to denote lists or symbols, the double quote is used only with character strings.
	In this book, the only use for character strings is as items to be printed.
}

Now we can distinguish between symbols and their values:

\begin{scheme}
  (define a 1)

  (define b 2)

  (list a b)
  ~\outprint{(1 2)}~

  (list 'a 'b)
  ~\outprint{(a b)}~

  (list 'a b)
  ~\outprint{(a 2)}~
\end{scheme}

Quotation also allows us to type in compound objects, using the conventional printed representation for lists:%
\footnote{
	Strictly, our use of the quotation mark violates the general rule that all compound expressions in our language should be delimited by parentheses and look like lists.
	We can recover this consistency by introducing a special form \code{quote}, which serves the same purpose as the quotation mark.
	Thus, we would type \code{(quote a)} instead of \code{'a}, and we would type \code{(quote (a b c))} instead of \code{'(a b c)}.
	This is precisely how the interpreter works.
	The quotation mark is just a single-character abbreviation for wrapping the next complete expression with \code{quote} to form \code{(quote ⟨\var{expression}⟩)}.
	This is important because it maintains the principle that any expression seen by the interpreter can be manipulated as a data object.
	For instance, we could construct the expression \code{(car '(a b c))}, which is the same as \code{(car (quote (a b c)))}, by evaluating \code{(list 'car (list 'quote '(a b c)))}.
}
\begin{scheme}
  (car '(a b c))
  ~\outprint{a}~

  (cdr '(a b c))
  ~\outprint{(b c)}~
\end{scheme}
In keeping with this, we can obtain the empty list by evaluating \code{'()}, and thus dispense with the variable \code{nil}.

One additional primitive used in manipulating symbols is \code{eq?}, which takes two symbols as arguments and tests whether they are the same.%
\footnote{
	We can consider two symbols to be “the same” if they consist of the same characters in the same order.
	Such a definition skirts a deep issue that we are not yet ready to address:
	the meaning of “sameness” in a programming language.
	We will return to this in \link{Chapter 3} (\link{Section 3.1.3}).
}
Using \code{eq?}, we can implement a useful procedure called \code{memq}.
This takes two arguments, a symbol and a list.
If the symbol is not contained in the list (i.e., is not \code{eq?} to any item in the list), then \code{memq} returns false.
Otherwise, it returns the sublist of the list beginning with the first occurrence of the symbol:
\begin{scheme}
  (define (memq item x)
    (cond ((null? x) false)
          ((eq? item (car x)) x)
          (else (memq item (cdr x)))))
\end{scheme}
For example, the value of
\begin{scheme}
  (memq 'apple '(pear banana prune))
\end{scheme}
is false, whereas the value of
\begin{scheme}
  (memq 'apple '(x (apple sauce) y apple pear))
\end{scheme}
is \code{(apple pear)}.



\begin{exercise}
	\label{Exercise 2.53}
	What would the interpreter print in response to evaluating each of the following expressions?
	\begin{scheme}
	  (list 'a 'b 'c)

	  (list (list 'george))

	  (cdr '((x1 x2) (y1 y2)))

	  (cadr '((x1 x2) (y1 y2)))

	  (pair? (car '(a short list)))

	  (memq 'red '((red shoes) (blue socks)))

	  (memq 'red '(red shoes blue socks))
	\end{scheme}
\end{exercise}



\begin{exercise}
	\label{Exercise 2.54}
	Two lists are said to be \code{equal?} if they contain equal elements arranged in the same order.
	For example,
	\begin{scheme}
	  (equal? '(this is a list) '(this is a list))
	\end{scheme}
	is true, but
	\begin{scheme}
	  (equal? '(this is a list) '(this (is a) list))
	\end{scheme}
	is false.
	To be more precise, we can define \code{equal?}  recursively in terms of the basic \code{eq?} equality of symbols by saying that \code{a} and \code{b} are \code{equal?} if they are both symbols and the symbols are \code{eq?}, or if they are both lists such that \code{(car a)} is \code{equal?} to \code{(car b)} and \code{(cdr a)} is \code{equal?} to \code{(cdr b)}.
	Using this idea, implement \code{equal?} as a procedure.%
	\footnote{
		In practice, programmers use \code{equal?} to compare lists that contain numbers as well as symbols.
		Numbers are not considered to be symbols.
		The question of whether two numerically equal numbers (as tested by \code{=}) are also \code{eq?} is highly implementation-dependent.
		A better definition of \code{equal?} (such as the one that comes as a primitive in Scheme) would also stipulate that if \code{a} and \code{b} are both numbers, then \code{a} and \code{b} are \code{equal?} if they are numerically equal.
	}
\end{exercise}



\begin{exercise}
	\label{Exercise 2.55}
	Eva Lu Ator types to the interpreter the expression
	\begin{scheme}
	  (car ''abracadabra)
	\end{scheme}
	To her surprise, the interpreter prints back \code{quote}.
	Explain.
\end{exercise}
