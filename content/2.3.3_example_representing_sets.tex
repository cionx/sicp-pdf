\subsection{Example: Representing Sets}
\label{Section 2.3.3}

In the previous examples we built representations for two kinds of compound data objects:
rational numbers and algebraic expressions.
In one of these examples we had the choice of simplifying (reducing) the expressions at either construction time or selection time, but other than that the choice of a representation for these structures in terms of lists was straightforward.
When we turn to the representation of sets, the choice of a representation is not so obvious.
Indeed, there are a number of possible representations, and they differ significantly from one another in several ways.

Informally, a set is simply a collection of distinct objects.
To give a more precise definition we can employ the method of data abstraction.
That is, we define “set” by specifying the operations that are to be used on sets.
These are \code{union-set}, \code{intersection-set}, \code{element-of-set?}, and \code{adjoin-set}.
\code{element-of-set?} is a predicate that determines whether a given element is a member of a set.
\code{adjoin-set} takes an object and a set as arguments and returns a set that contains the elements of the original set and also the adjoined element.
\code{union-set} computes the union of two sets, which is the set containing each element that appears in either argument.
\code{intersection-set} computes the intersection of two sets, which is the set containing only elements that appear in both arguments.
From the viewpoint of data abstraction, we are free to design any representation that implements these operations in a way consistent with the interpretations given above.%
\footnote{
	If we want to be more formal, we can specify “consistent with the interpretations given above” to mean that the operations satisfy a collection of rules such as these:
	\begin{itemize}

		\item
			For any set \code{S} and any object \code{x}, \code{(element-of-set? x (adjoin-set x S))} is true
			(informally: “Adjoining an object to a set produces a set that contains the object”).

		\item
			For any sets \code{S} and \code{T} and any object \code{x}, \code{(element-of-set? x (union-set S T))} is equal to \code{(or (element-of-set? x S) (element-of-set? x T))}
			(informally: “The elements of \code{(union S T)} are the elements that are in \code{S} or in \code{T}”).

		\item
			For any object \code{x}, \code{(element-of-set? x '())} is false
			(informally: “No object is an element of the empty set”).
	\end{itemize}
}



\subsubsection*{Sets as unordered lists}

One way to represent a set is as a list of its elements in which no element appears more than once.
The empty set is represented by the empty list.
In this representation, \code{element-of-set?} is similar to the procedure \code{memq} of \cref{Section 2.3.1}.
It uses \code{equal?} instead of \code{eq?} so that the set elements need not be symbols:
\begin{scheme}
  (define (element-of-set? x set)
    (cond ((null? set) false)
          ((equal? x (car set)) true)
          (else (element-of-set? x (cdr set)))))
\end{scheme}
Using this, we can write \code{adjoin-set}.
If the object to be adjoined is already in the set, we just return the set.
Otherwise, we use \code{cons} to add the object to the list that represents the set:
\begin{scheme}
  (define (adjoin-set x set)
    (if (element-of-set? x set)
        set
        (cons x set)))
\end{scheme}
For \code{intersection-set} we can use a recursive strategy.
If we know how to form the intersection of \code{set2} and the \code{cdr} of \code{set1}, we only need to decide whether to include the \code{car} of \code{set1} in this.
But this depends on whether \code{(car set1)} is also in \code{set2}.
Here is the resulting procedure:
\begin{scheme}
  (define (intersection-set set1 set2)
    (cond ((or (null? set1) (null? set2)) '())
          ((element-of-set? (car set1) set2)
           (cons (car set1)
                 (intersection-set (cdr set1) set2)))
          (else (intersection-set (cdr set1) set2))))
\end{scheme}

In designing a representation, one of the issues we should be concerned with is efficiency.
Consider the number of steps required by our set operations.
Since they all use \code{element-of-set?}, the speed of this operation has a major impact on the efficiency of the set implementation as a whole.
Now, in order to check whether an object is a member of a set, \code{element-of-set?} may have to scan the entire set.
(In the worst case, the object turns out not to be in the set.)
Hence, if the set has \( n \) elements, \code{element-of-set?} might take up to \( n \) steps.
Thus, the number of steps required grows as \( Θ(n) \).
The number of steps required by \code{adjoin-set}, which uses this operation, also grows as \( Θ(n) \).
For \code{intersection-set}, which does an \code{element-of-set?} check for each element of \code{set1}, the number of steps required grows as the product of the sizes of the sets involved, or \( Θ(n^2) \) for two sets of size \( n \).
The same will be true of \code{union-set}.



\begin{exercise}
	\label{Exercise 2.59}
	Implement the \code{union-set} operation for the unordered-list representation of sets.
\end{exercise}



\begin{exercise}
	\label{Exercise 2.60}
	We specified that a set would be represented as a list with no duplicates.
	Now suppose we allow duplicates.
	For instance, the set \( \{1, 2, 3\} \) could be represented as the list \code{(2 3 2 1 3 2 2)}.
	Design procedures \code{element-of-set?}, \code{adjoin-set}, \code{union-set}, and \code{intersection-set} that operate on this representation.
	How does the efficiency of each compare with the corresponding procedure for the non-duplicate representation?
	Are there applications for which you would use this representation in preference to the non-duplicate one?
\end{exercise}



\subsubsection*{Sets as ordered lists}

One way to speed up our set operations is to change the representation so that the set elements are listed in increasing order.
To do this, we need some way to compare two objects so that we can say which is bigger.
For example, we could compare symbols lexicographically, or we could agree on some method for assigning a unique number to an object and then compare the elements by comparing the corresponding numbers.
To keep our discussion simple, we will consider only the case where the set elements are numbers, so that we can compare elements using \code{>} and \code{<}.
We will represent a set of numbers by listing its elements in increasing order.
Whereas our first representation above allowed us to represent the set \( \{1, 3, 6, 10\} \) by listing the elements in any order, our new representation allows only the list \code{(1 3 6 10)}.

One advantage of ordering shows up in \code{element-of-set?}:
In checking for the presence of an item, we no longer have to scan the entire set.
If we reach a set element that is larger than the item we are looking for, then we know that the item is not in the set:
\begin{scheme}
  (define (element-of-set? x set)
    (cond ((null? set) false)
          ((= x (car set)) true)
          ((< x (car set)) false)
          (else (element-of-set? x (cdr set)))))
\end{scheme}
How many steps does this save?
In the worst case, the item we are looking for may be the largest one in the set, so the number of steps is the same as for the unordered representation.
On the other hand, if we search for items of many different sizes we can expect that sometimes we will be able to stop searching at a point near the beginning of the list and that other times we will still need to examine most of the list.
On the average we should expect to have to examine about half of the items in the set.
Thus, the average number of steps required will be about \( n / 2 \).
This is still \( Θ(n) \) growth, but it does save us, on the average, a factor of \( 2 \) in number of steps over the previous implementation.

We obtain a more impressive speedup with \code{intersection-set}.
In the unordered representation this operation required \( Θ(n^2) \) steps, because we performed a complete scan of \code{set2} for each element of \code{set1}.
But with the ordered representation, we can use a more clever method.
Begin by comparing the initial elements, \code{x1} and \code{x2}, of the two sets.
If \code{x1} equals \code{x2}, then that gives an element of the intersection, and the rest of the intersection is the intersection of the \code{cdr}-s of the two sets.
Suppose, however, that \code{x1} is less than \code{x2}.
Since \code{x2} is the smallest element in \code{set2}, we can immediately conclude that \code{x1} cannot appear anywhere in \code{set2} and hence is not in the intersection.
Hence, the intersection is equal to the intersection of \code{set2} with the \code{cdr} of \code{set1}.
Similarly, if \code{x2} is less than \code{x1}, then the intersection is given by the intersection of \code{set1} with the \code{cdr} of \code{set2}.
Here is the procedure:
\begin{scheme}
  (define (intersection-set set1 set2)
    (if (or (null? set1) (null? set2))
        '()
        (let ((x1 (car set1)) (x2 (car set2)))
          (cond ((= x1 x2)
                 (cons x1 (intersection-set (cdr set1)
                                            (cdr set2))))
                ((< x1 x2)
                 (intersection-set (cdr set1) set2))
                ((< x2 x1)
                 (intersection-set set1 (cdr set2)))))))
\end{scheme}
To estimate the number of steps required by this process, observe that at each step we reduce the intersection problem to computing intersections of smaller sets---removing the first element from \code{set1} or \code{set2} or both.
Thus, the number of steps required is at most the sum of the sizes of \code{set1} and \code{set2}, rather than the product of the sizes as with the unordered representation.
This is \( Θ(n) \) growth rather than \( Θ(n^2) \)---a considerable speedup, even for sets of moderate size.



\begin{exercise}
	\label{Exercise 2.61}
	Give an implementation of \code{adjoin-set} using the ordered representation.
	By analogy with \code{element-of-set?} show how to take advantage of the ordering to produce a procedure that requires on the average about half as many steps as with the unordered representation.
\end{exercise}



\begin{exercise}
	\label{Exercise 2.62}
	Give a \( Θ(n) \) implementation of \code{union-set} for sets represented as ordered lists.
\end{exercise}



\subsubsection*{Sets as binary trees}

We can do better than the ordered-list representation by arranging the set elements in the form of a tree.
Each node of the tree holds one element of the set, called the “entry” at that node, and a link to each of two other (possibly empty) nodes.
The “left” link points to elements smaller than the one at the node, and the “right” link to elements greater than the one at the node.
\cref{Figure 2.16} shows some trees that represent the set \( \{ 1, 3, 5, 7, 9, 11 \} \).
The same set may be represented by a tree in a number of different ways.
The only thing we require for a valid representation is that all elements in the left subtree be smaller than the node entry and that all elements in the right subtree be larger.

\begin{figure}[tb]
	\centering
	\includesvg[width=70mm]{fig/chap2/Fig2.16b.svg}
	\caption{
		Various binary trees that represent the set \( \{ 1, 3, 5, 7, 9, 11 \} \).
	}
	\label{Figure 2.16}
\end{figure}

The advantage of the tree representation is this:
Suppose we want to check whether a number \( x \) is contained in a set.
We begin by comparing \( x \) with the entry in the top node.
If \( x \) is less than this, we know that we need only search the left subtree;
if \( x \) is greater, we need only search the right subtree.
Now, if the tree is “balanced,” each of these subtrees will be about half the size of the original.
Thus, in one step we have reduced the problem of searching a tree of size \( n \) to searching a tree of size \( n / 2 \).
Since the size of the tree is halved at each step, we should expect that the number of steps needed to search a tree of size \( n \) grows as \( Θ(\log n) \).%
\footnote{
	Halving the size of the problem at each step is the distinguishing characteristic of logarithmic growth, as we saw with the fast-exponentiation algorithm of \cref{Section 1.2.4} and the half-interval search method of \cref{Section 1.3.3}.
}
For large sets, this will be a significant speedup over the previous representations.

We can represent trees by using lists.
Each node will be a list of three items:
the entry at the node, the left subtree, and the right subtree.
A left or a right subtree of the empty list will indicate that there is no subtree connected there.
We can describe this representation by the following procedures:%
\footnote{
	We are representing sets in terms of trees, and trees in terms of lists---in effect, a data abstraction built upon a data abstraction.
	We can regard the procedures \code{entry}, \code{left-branch}, \code{right-branch}, and \code{make-tree} as a way of isolating the abstraction of a “binary tree” from the particular way we might wish to represent such a tree in terms of list structure.
}
\begin{scheme}
  (define (entry tree) (car tree))

  (define (left-branch tree) (cadr tree))

  (define (right-branch tree) (caddr tree))

  (define (make-tree entry left right)
    (list entry left right))
\end{scheme}

Now we can write the \code{element-of-set?} procedure using the strategy described above:
\begin{scheme}
  (define (element-of-set? x set)
    (cond ((null? set) false)
          ((= x (entry set)) true)
          ((< x (entry set))
           (element-of-set? x (left-branch set)))
          ((> x (entry set))
           (element-of-set? x (right-branch set)))))
\end{scheme}

Adjoining an item to a set is implemented similarly and also requires \( Θ(\log n) \) steps.
To adjoin an item \code{x}, we compare \code{x} with the node entry to determine whether \code{x} should be added to the right or to the left branch, and having adjoined \code{x} to the appropriate branch we piece this newly constructed branch together with the original entry and the other branch.
If \code{x} is equal to the entry, we just return the node.
If we are asked to adjoin \code{x} to an empty tree, we generate a tree that has \code{x} as the entry and empty right and left branches.
Here is the procedure:
\begin{scheme}
  (define (adjoin-set x set)
    (cond ((null? set) (make-tree x '() '()))
          ((= x (entry set)) set)
          ((< x (entry set))
           (make-tree (entry set)
                      (adjoin-set x (left-branch set))
                      (right-branch set)))
          ((> x (entry set))
           (make-tree (entry set)
                      (left-branch set)
                      (adjoin-set x (right-branch set))))))
\end{scheme}

The above claim that searching the tree can be performed in a logarithmic number of steps rests on the assumption that the tree is “balanced,” i.e., that the left and the right subtree of every tree have approximately the same number of elements, so that each subtree contains about half the elements of its parent.
But how can we be certain that the trees we construct will be balanced?
Even if we start with a balanced tree, adding elements with \code{adjoin-set} may produce an unbalanced result.
Since the position of a newly adjoined element depends on how the element compares with the items already in the set, we can expect that if we add elements “randomly” the tree will tend to be balanced on the average.
But this is not a guarantee.
For example, if we start with an empty set and adjoin the numbers 1 through 7 in sequence we end up with the highly unbalanced tree shown in \cref{Figure 2.17}.
In this tree all the left subtrees are empty, so it has no advantage over a simple ordered list.
One way to solve this problem is to define an operation that transforms an arbitrary tree into a balanced tree with the same elements.
Then we can perform this transformation after every few \code{adjoin-set} operations to keep our set in balance.
There are also other ways to solve this problem, most of which involve designing new data structures for which searching and insertion both can be done in \( Θ(\log n) \) steps.%
\footnote{
	Examples of such structures include \newterm{B-trees} and \newterm{red-black trees}.
	There is a large literature on data structures devoted to this problem.
	See \cref{Cormen et al. 1990}.
}

\begin{figure}[tb]
	\centering
	\includesvg[width=40mm]{fig/chap2/Fig2.17a.svg}
	\caption{
		Unbalanced tree produced by adjoining \( 1 \) through \( 7 \) in sequence.
	}
	\label{Figure 2.17}
\end{figure}



\begin{exercise}
	\label{Exercise 2.63}
	Each of the following two procedures converts a binary tree to a list.
	\begin{scheme}
	  (define (tree->list-1 tree)
	    (if (null? tree)
	        '()
	        (append (tree->list-1 (left-branch tree))
	                (cons (entry tree)
	                      (tree->list-1 (right-branch tree))))))

	  (define (tree->list-2 tree)
	    (define (copy-to-list tree result-list)
	      (if (null? tree)
	          result-list
	          (copy-to-list (left-branch tree)
	                        (cons (entry tree)
	                              (copy-to-list (right-branch tree)
	                                            result-list)))))
	    (copy-to-list tree '()))
	\end{scheme}
	\begin{enumerate}[label = \alph*., leftmargin = *]

		\item
			Do the two procedures produce the same result for every tree?
			If not, how do the results differ?
			What lists do the two procedures produce for the trees in \cref{Figure 2.16}?

		\item
			Do the two procedures have the same order of growth in the number of steps required to convert a balanced tree with \( n \) elements to a list?
			If not, which one grows more slowly?

	\end{enumerate}
\end{exercise}



\begin{exercise}
	\label{Exercise 2.64}
	The following procedure \code{list->tree} converts an ordered list to a balanced binary tree.
	The helper procedure \code{partial-tree} takes as arguments an integer \( n \) and list of at least \( n \) elements and constructs a balanced tree containing the first \( n \) elements of the list.
	The result returned by \code{partial-tree} is a pair (formed with \code{cons}) whose \code{car} is the constructed tree and whose \code{cdr} is the list of elements not included in the tree.
	\begin{scheme}
	  (define (list->tree elements)
	    (car (partial-tree elements (length elements))))

	  (define (partial-tree elts n)
	    (if (= n 0)
	        (cons '() elts)
	        (let ((left-size (quotient (- n 1) 2)))
	          (let ((left-result
	                 (partial-tree elts left-size)))
	            (let ((left-tree (car left-result))
	                  (non-left-elts (cdr left-result))
	                  (right-size (- n (+ left-size 1))))
	              (let ((this-entry (car non-left-elts))
	                    (right-result
	                     (partial-tree
	                      (cdr non-left-elts)
	                      right-size)))
	                (let ((right-tree (car right-result))
	                      (remaining-elts
	                       (cdr right-result)))
	                  (cons (make-tree this-entry
	                                   left-tree
	                                   right-tree)
	                        remaining-elts))))))))
	\end{scheme}
	\begin{enumerate}[label = \alph*., leftmargin = *]

		\item
			Write a short paragraph explaining as clearly as you can how \code{partial-tree} works.
			Draw the tree produced by \code{list->tree} for the list \code{(1 3 5 7 9 11)}.

		\item
			What is the order of growth in the number of steps required by \code{list->tree} to convert a list of \( n \) elements?

	\end{enumerate}
\end{exercise}



\begin{exercise}
	\label{Exercise 2.65}
	Use the results of \cref{Exercise 2.63} and \cref{Exercise 2.64} to give \( Θ(n) \) implementations of \code{union-set} and \code{intersection-set} for sets implemented as (balanced) binary trees.%
	\footnote{
		\cref{Exercise 2.63} through \cref{Exercise 2.65} are due to Paul Hilfinger.
	}
\end{exercise}



\subsubsection*{Sets and information retrieval}

We have examined options for using lists to represent sets and have seen how the choice of representation for a data object can have a large impact on the performance of the programs that use the data.
Another reason for concentrating on sets is that the techniques discussed here appear again and again in applications involving information retrieval.

Consider a data base containing a large number of individual records, such as the personnel files for a company or the transactions in an accounting system.
A typical data-management system spends a large amount of time accessing or modifying the data in the records and therefore requires an efficient method for accessing records.
This is done by identifying a part of each record to serve as an identifying \newterm{key}.
A key can be anything that uniquely identifies the record.
For a personnel file, it might be an employee’s \acronym{ID} number.
For an accounting system, it might be a transaction number.
Whatever the key is, when we define the record as a data structure we should include a \code{key} selector procedure that retrieves the key associated with a given record.

Now we represent the data base as a set of records.
To locate the record with a given key we use a procedure \code{lookup}, which takes as arguments a key and a data base and which returns the record that has that key, or false if there is no such record.
\code{lookup} is implemented in almost the same way as \code{element-of-set?}.
For example, if the set of records is implemented as an unordered list, we could use
\begin{scheme}
  (define (lookup given-key set-of-records)
    (cond ((null? set-of-records) false)
          ((equal? given-key (key (car set-of-records)))
           (car set-of-records))
          (else (lookup given-key (cdr set-of-records)))))
\end{scheme}

Of course, there are better ways to represent large sets than as unordered lists.
Information-retrieval systems in which records have to be “randomly accessed” are typically implemented by a tree-based method, such as the binary-tree representation discussed previously.
In designing such a system the methodology of data abstraction can be a great help.
The designer can create an initial implementation using a simple, straightforward representation such as unordered lists.
This will be unsuitable for the eventual system, but it can be useful in providing a “quick and dirty” data base with which to test the rest of the system.
Later on, the data representation can be modified to be more sophisticated.
If the data base is accessed in terms of abstract selectors and constructors, this change in representation will not require any changes to the rest of the system.

\begin{exercise}
	\label{Exercise 2.66}
	Implement the \code{lookup} procedure for the case where the set of records is structured as a binary tree, ordered by the numerical values of the keys.
\end{exercise}
