\subsection{Evaluator Data Structures}
\label{Section 4.1.3}

In addition to defining the external syntax of expressions, the evaluator implementation must also define the data structures that the evaluator manipulates internally, as part of the execution of a program, such as the representation of procedures and environments and the representation of true and false.



\subsubsection*{Testing of predicates}

For conditionals, we accept anything to be true that is not the explicit \code{false} object.
\begin{scheme}
  (define (true? x)  (not (eq? x false)))

  (define (false? x) (eq? x false))
\end{scheme}



\subsubsection*{Representing procedures}

To handle primitives, we assume that we have available the following procedures:
\begin{itemize}

	\item
		\code{(apply-primitive-procedure ⟨\var{proc}⟩ ⟨\var{args}⟩)} \\
		applies the given primitive procedure to the argument values in the list \code{⟨\var{args}⟩} and returns the result of the application.

	\item
		\code{(primitive-procedure?  ⟨\var{proc}⟩)} \\
		tests whether \code{⟨\var{proc}⟩} is a primitive procedure.

\end{itemize}
These mechanisms for handling primitives are further described in \cref{Section 4.1.4}.

Compound procedures are constructed from parameters, procedure bodies, and environments using the constructor \code{make-procedure}:
\begin{scheme}
  (define (make-procedure parameters body env)
    (list 'procedure parameters body env))

  (define (compound-procedure? p)
    (tagged-list? p 'procedure))

  (define (procedure-parameters p) (cadr p))

  (define (procedure-body p) (caddr p))

  (define (procedure-environment p) (cadddr p))
\end{scheme}



\subsubsection*{Operations on Environments}

The evaluator needs operations for manipulating environments.
As explained in \cref{Section 3.2}, an environment is a sequence of frames, where each frame is a table of bindings that associate variables with their corresponding values.
We use the following operations for manipulating environments:
\begin{itemize}

	\item
		\code{(lookup-variable-value ⟨\var{var}⟩ ⟨\var{env}⟩)} \\
		returns the value that is bound to the symbol \code{⟨\var{var}⟩} in the environment \code{⟨\var{env}⟩}, or signals an error if the variable is unbound.

	\item
		\code{(extend-environment ⟨\var{variables}⟩ ⟨\var{values}⟩ ⟨\var{base-env}⟩)} \\
		returns a new environment, consisting of a new frame in which the symbols in the list \code{⟨\var{variables}⟩} are bound to the corresponding elements in the list \code{⟨\var{values}⟩}, where the enclosing environment is the environment ⟨\var{base-env}⟩.

	\item
		\code{(define-variable! ⟨\var{var}⟩ ⟨\var{value}⟩ ⟨\var{env}⟩)} \\
		adds to the first frame in the environment \code{⟨\var{env}⟩} a new binding that associates the variable \code{⟨\var{var}⟩} with the value \code{⟨\var{value}⟩}.

	\item
		\code{(set-variable-value! ⟨\var{var}⟩ ⟨\var{value}⟩ ⟨\var{env}⟩)} \\
		changes the binding of the variable \code{⟨\var{var}⟩} in the environment \code{⟨\var{env}⟩} so that the variable is now bound to the value \code{⟨\var{value}⟩}, or signals an error if the variable is unbound.

\end{itemize}

To implement these operations we represent an environment as a list of frames.
The enclosing environment of an environment is the \code{cdr} of the list.
The empty environment is simply the empty list.
\begin{scheme}
  (define (enclosing-environment env) (cdr env))

  (define (first-frame env) (car env))

  (define the-empty-environment '())
\end{scheme}

Each frame of an environment is represented as a pair of lists:
a list of the variables bound in that frame and a list of the associated values.%
\footnote{
	Frames are not really a data abstraction in the following code:
	\code{set-variable-value!} and \code{define-variable!} use \code{set-car!} to directly modify the values in a frame.
	The purpose of the frame procedures is to make the environment-manipulation procedures easy to read.
}
\begin{scheme}
  (define (make-frame variables values)
    (cons variables values))

  (define (frame-variables frame) (car frame))

  (define (frame-values frame) (cdr frame))

  (define (add-binding-to-frame! var val frame)
    (set-car! frame (cons var (car frame)))
    (set-cdr! frame (cons val (cdr frame))))
\end{scheme}

To extend an environment by a new frame that associates variables with values, we make a frame consisting of the list of variables and the list of values, and we adjoin this to the environment.
We signal an error if the number of variables does not match the number of values.
\begin{scheme}
  (define (extend-environment vars vals base-env)
    (if (= (length vars) (length vals))
        (cons (make-frame vars vals) base-env)
        (if (< (length vars) (length vals))
            (error "Too many arguments supplied" vars vals)
            (error "Too few arguments supplied" vars vals))))
\end{scheme}

To look up a variable in an environment, we scan the list of variables in the first frame.
If we find the desired variable, we return the corresponding element in the list of values.
If we do not find the variable in the current frame, we search the enclosing environment, and so on.
If we reach the empty environment, we signal an “unbound variable” error.
\begin{scheme}
  (define (lookup-variable-value var env)
    (define (env-loop env)
      (define (scan vars vals)
        (cond ((null? vars)
               (env-loop (enclosing-environment env)))
              ((eq? var (car vars)) (car vals))
              (else (scan (cdr vars) (cdr vals)))))
      (if (eq? env the-empty-environment)
          (error "Unbound variable" var)
          (let ((frame (first-frame env)))
            (scan (frame-variables frame)
                  (frame-values frame)))))
    (env-loop env))
\end{scheme}

To set a variable to a new value in a specified environment, we scan for the variable, just as in \code{lookup-variable-value}, and change the corresponding value when we find it.
\begin{scheme}
  (define (set-variable-value! var val env)
    (define (env-loop env)
      (define (scan vars vals)
        (cond ((null? vars)
               (env-loop (enclosing-environment env)))
              ((eq? var (car vars)) (set-car! vals val))
              (else (scan (cdr vars) (cdr vals)))))
      (if (eq? env the-empty-environment)
          (error "Unbound variable: SET!" var)
          (let ((frame (first-frame env)))
            (scan (frame-variables frame)
                  (frame-values frame)))))
    (env-loop env))
\end{scheme}

To define a variable, we search the first frame for a binding for the variable, and change the binding if it exists (just as in \code{set-variable-value!}).
If no such binding exists, we adjoin one to the first frame.
\begin{scheme}
  (define (define-variable! var val env)
    (let ((frame (first-frame env)))
      (define (scan vars vals)
        (cond ((null? vars)
               (add-binding-to-frame! var val frame))
              ((eq? var (car vars)) (set-car! vals val))
              (else (scan (cdr vars) (cdr vals)))))
      (scan (frame-variables frame) (frame-values frame))))
\end{scheme}

The method described here is only one of many plausible ways to represent environments.
Since we used data abstraction to isolate the rest of the evaluator from the detailed choice of representation, we could change the environment representation if we wanted to.
(See \cref{Exercise 4.11}.)
In a production-quality Lisp system, the speed of the evaluator’s environment operations---especially that of variable lookup---has a major impact on the performance of the system.
The representation described here, although conceptually simple, is not efficient and would not ordinarily be used in a production system.%
\footnote{
	The drawback of this representation (as well as the variant in \cref{Exercise 4.11}) is that the evaluator may have to search through many frames in order to find the binding for a given variable.
	(Such an approach is referred to as \newterm{deep binding}.)
	One way to avoid this inefficiency is to make use of a strategy called \newterm{lexical addressing}, which will be discussed in \cref{Section 5.5.6}.
}



\begin{exercise}
	\label{Exercise 4.11}
	Instead of representing a frame as a pair of lists, we can represent a frame as a list of bindings, where each binding is a name-value pair.
	Rewrite the environment operations to use this alternative representation.
\end{exercise}



\begin{exercise}
	\label{Exercise 4.12}
	The procedures \code{set-variable-value!}, \code{define-variable!} and \code{lookup-variable-value} can be expressed in terms of more abstract procedures for traversing the environment structure.
	Define abstractions that capture the common patterns and redefine the three procedures in terms of these abstractions.
\end{exercise}



\begin{exercise}
	\label{Exercise 4.13}
	Scheme allows us to create new bindings for variables by means of \code{define}, but provides no way to get rid of bindings.
	Implement for the evaluator a special form \code{make-unbound!} that removes the binding of a given symbol from the environment in which the \code{make-unbound!} expression is evaluated.
	This problem is not completely specified.
	For example, should we remove only the binding in the first frame of the environment?
	Complete the specification and justify any choices you make.
\end{exercise}
